\documentclass[11pt, a4paper, fleqn]{article}
\usepackage{cp2526t}
\makeindex

%================= lhs2tex=====================================================%
%% ODER: format ==         = "\mathrel{==}"
%% ODER: format /=         = "\neq "
%
%
\makeatletter
\@ifundefined{lhs2tex.lhs2tex.sty.read}%
  {\@namedef{lhs2tex.lhs2tex.sty.read}{}%
   \newcommand\SkipToFmtEnd{}%
   \newcommand\EndFmtInput{}%
   \long\def\SkipToFmtEnd#1\EndFmtInput{}%
  }\SkipToFmtEnd

\newcommand\ReadOnlyOnce[1]{\@ifundefined{#1}{\@namedef{#1}{}}\SkipToFmtEnd}
\usepackage{amstext}
\usepackage{amssymb}
\usepackage{stmaryrd}
\DeclareFontFamily{OT1}{cmtex}{}
\DeclareFontShape{OT1}{cmtex}{m}{n}
  {<5><6><7><8>cmtex8
   <9>cmtex9
   <10><10.95><12><14.4><17.28><20.74><24.88>cmtex10}{}
\DeclareFontShape{OT1}{cmtex}{m}{it}
  {<-> ssub * cmtt/m/it}{}
\newcommand{\texfamily}{\fontfamily{cmtex}\selectfont}
\DeclareFontShape{OT1}{cmtt}{bx}{n}
  {<5><6><7><8>cmtt8
   <9>cmbtt9
   <10><10.95><12><14.4><17.28><20.74><24.88>cmbtt10}{}
\DeclareFontShape{OT1}{cmtex}{bx}{n}
  {<-> ssub * cmtt/bx/n}{}
\newcommand{\tex}[1]{\text{\texfamily#1}}	% NEU

\newcommand{\Sp}{\hskip.33334em\relax}


\newcommand{\Conid}[1]{\mathit{#1}}
\newcommand{\Varid}[1]{\mathit{#1}}
\newcommand{\anonymous}{\kern0.06em \vbox{\hrule\@width.5em}}
\newcommand{\plus}{\mathbin{+\!\!\!+}}
\newcommand{\bind}{\mathbin{>\!\!\!>\mkern-6.7mu=}}
\newcommand{\rbind}{\mathbin{=\mkern-6.7mu<\!\!\!<}}% suggested by Neil Mitchell
\newcommand{\sequ}{\mathbin{>\!\!\!>}}
\renewcommand{\leq}{\leqslant}
\renewcommand{\geq}{\geqslant}
\usepackage{polytable}

%mathindent has to be defined
\@ifundefined{mathindent}%
  {\newdimen\mathindent\mathindent\leftmargini}%
  {}%

\def\resethooks{%
  \global\let\SaveRestoreHook\empty
  \global\let\ColumnHook\empty}
\newcommand*{\savecolumns}[1][default]%
  {\g@addto@macro\SaveRestoreHook{\savecolumns[#1]}}
\newcommand*{\restorecolumns}[1][default]%
  {\g@addto@macro\SaveRestoreHook{\restorecolumns[#1]}}
\newcommand*{\aligncolumn}[2]%
  {\g@addto@macro\ColumnHook{\column{#1}{#2}}}

\resethooks

\newcommand{\onelinecommentchars}{\quad-{}- }
\newcommand{\commentbeginchars}{\enskip\{-}
\newcommand{\commentendchars}{-\}\enskip}

\newcommand{\visiblecomments}{%
  \let\onelinecomment=\onelinecommentchars
  \let\commentbegin=\commentbeginchars
  \let\commentend=\commentendchars}

\newcommand{\invisiblecomments}{%
  \let\onelinecomment=\empty
  \let\commentbegin=\empty
  \let\commentend=\empty}

\visiblecomments

\newlength{\blanklineskip}
\setlength{\blanklineskip}{0.66084ex}

\newcommand{\hsindent}[1]{\quad}% default is fixed indentation
\let\hspre\empty
\let\hspost\empty
\newcommand{\NB}{\textbf{NB}}
\newcommand{\Todo}[1]{$\langle$\textbf{To do:}~#1$\rangle$}

\EndFmtInput
\makeatother
%
%
%
%
%
%
% This package provides two environments suitable to take the place
% of hscode, called "plainhscode" and "arrayhscode". 
%
% The plain environment surrounds each code block by vertical space,
% and it uses \abovedisplayskip and \belowdisplayskip to get spacing
% similar to formulas. Note that if these dimensions are changed,
% the spacing around displayed math formulas changes as well.
% All code is indented using \leftskip.
%
% Changed 19.08.2004 to reflect changes in colorcode. Should work with
% CodeGroup.sty.
%
\ReadOnlyOnce{polycode.fmt}%
\makeatletter

\newcommand{\hsnewpar}[1]%
  {{\parskip=0pt\parindent=0pt\par\vskip #1\noindent}}

% can be used, for instance, to redefine the code size, by setting the
% command to \small or something alike
\newcommand{\hscodestyle}{}

% The command \sethscode can be used to switch the code formatting
% behaviour by mapping the hscode environment in the subst directive
% to a new LaTeX environment.

\newcommand{\sethscode}[1]%
  {\expandafter\let\expandafter\hscode\csname #1\endcsname
   \expandafter\let\expandafter\endhscode\csname end#1\endcsname}

% "compatibility" mode restores the non-polycode.fmt layout.

\newenvironment{compathscode}%
  {\par\noindent
   \advance\leftskip\mathindent
   \hscodestyle
   \let\\=\@normalcr
   \let\hspre\(\let\hspost\)%
   \pboxed}%
  {\endpboxed\)%
   \par\noindent
   \ignorespacesafterend}

\newcommand{\compaths}{\sethscode{compathscode}}

% "plain" mode is the proposed default.
% It should now work with \centering.
% This required some changes. The old version
% is still available for reference as oldplainhscode.

\newenvironment{plainhscode}%
  {\hsnewpar\abovedisplayskip
   \advance\leftskip\mathindent
   \hscodestyle
   \let\hspre\(\let\hspost\)%
   \pboxed}%
  {\endpboxed%
   \hsnewpar\belowdisplayskip
   \ignorespacesafterend}

\newenvironment{oldplainhscode}%
  {\hsnewpar\abovedisplayskip
   \advance\leftskip\mathindent
   \hscodestyle
   \let\\=\@normalcr
   \(\pboxed}%
  {\endpboxed\)%
   \hsnewpar\belowdisplayskip
   \ignorespacesafterend}

% Here, we make plainhscode the default environment.

\newcommand{\plainhs}{\sethscode{plainhscode}}
\newcommand{\oldplainhs}{\sethscode{oldplainhscode}}
\plainhs

% The arrayhscode is like plain, but makes use of polytable's
% parray environment which disallows page breaks in code blocks.

\newenvironment{arrayhscode}%
  {\hsnewpar\abovedisplayskip
   \advance\leftskip\mathindent
   \hscodestyle
   \let\\=\@normalcr
   \(\parray}%
  {\endparray\)%
   \hsnewpar\belowdisplayskip
   \ignorespacesafterend}

\newcommand{\arrayhs}{\sethscode{arrayhscode}}

% The mathhscode environment also makes use of polytable's parray 
% environment. It is supposed to be used only inside math mode 
% (I used it to typeset the type rules in my thesis).

\newenvironment{mathhscode}%
  {\parray}{\endparray}

\newcommand{\mathhs}{\sethscode{mathhscode}}

% texths is similar to mathhs, but works in text mode.

\newenvironment{texthscode}%
  {\(\parray}{\endparray\)}

\newcommand{\texths}{\sethscode{texthscode}}

% The framed environment places code in a framed box.

\def\codeframewidth{\arrayrulewidth}
\RequirePackage{calc}

\newenvironment{framedhscode}%
  {\parskip=\abovedisplayskip\par\noindent
   \hscodestyle
   \arrayrulewidth=\codeframewidth
   \tabular{@{}|p{\linewidth-2\arraycolsep-2\arrayrulewidth-2pt}|@{}}%
   \hline\framedhslinecorrect\\{-1.5ex}%
   \let\endoflinesave=\\
   \let\\=\@normalcr
   \(\pboxed}%
  {\endpboxed\)%
   \framedhslinecorrect\endoflinesave{.5ex}\hline
   \endtabular
   \parskip=\belowdisplayskip\par\noindent
   \ignorespacesafterend}

\newcommand{\framedhslinecorrect}[2]%
  {#1[#2]}

\newcommand{\framedhs}{\sethscode{framedhscode}}

% The inlinehscode environment is an experimental environment
% that can be used to typeset displayed code inline.

\newenvironment{inlinehscode}%
  {\(\def\column##1##2{}%
   \let\>\undefined\let\<\undefined\let\\\undefined
   \newcommand\>[1][]{}\newcommand\<[1][]{}\newcommand\\[1][]{}%
   \def\fromto##1##2##3{##3}%
   \def\nextline{}}{\) }%

\newcommand{\inlinehs}{\sethscode{inlinehscode}}

% The joincode environment is a separate environment that
% can be used to surround and thereby connect multiple code
% blocks.

\newenvironment{joincode}%
  {\let\orighscode=\hscode
   \let\origendhscode=\endhscode
   \def\endhscode{\def\hscode{\endgroup\def\@currenvir{hscode}\\}\begingroup}
   %\let\SaveRestoreHook=\empty
   %\let\ColumnHook=\empty
   %\let\resethooks=\empty
   \orighscode\def\hscode{\endgroup\def\@currenvir{hscode}}}%
  {\origendhscode
   \global\let\hscode=\orighscode
   \global\let\endhscode=\origendhscode}%

\makeatother
\EndFmtInput
%
%%format (bin (n) (k)) = "\Big(\vcenter{\xymatrix@R=1pt{" n "\\" k "}}\Big)"
%------------------------------------------------------------------------------%


%====== DEFINIR GRUPO E ELEMENTOS =============================================%

\group{G99}
\studentA{xxxxxx}{Nome }
\studentB{xxxxxx}{Nome }
\studentC{xxxxxx}{Nome }

%==============================================================================%

\begin{document}
\sffamily
\setlength{\parindent}{0em}
\emergencystretch 3em
\renewcommand{\baselinestretch}{1.25} 
\input{Cover}
\pagestyle{pagestyle}

\newgeometry{left=25mm,right=20mm,top=25mm,bottom=25mm}
\setlength{\parindent}{1em}

\section*{Preâmbulo}

Em \CP\ pretende-se ensinar a progra\-mação de computadores
como uma disciplina científica. Para isso parte-se de um repertório de \emph{combinadores}
que formam uma álgebra da programação % (conjunto de leis universais e seus corolários)
e usam-se esses combinadores para construir programas \emph{composicionalmente},
isto é, agregando programas já existentes.

Na sequência pedagógica dos planos de estudo dos cursos que têm
esta disciplina, opta-se pela aplicação deste método à programação
em \Haskell\ (sem prejuízo da sua aplicação a outras linguagens
funcionais). Assim, o presente trabalho prático coloca os
alunos perante problemas concretos que deverão ser implementados em
\Haskell. Há ainda um outro objectivo: o de ensinar a documentar
programas, a validá-los e a produzir textos técnico-científicos de
qualidade.

Antes de abordarem os problemas propostos no trabalho, os grupos devem ler
com atenção o anexo \ref{sec:documentacao} onde encontrarão as instruções
relativas ao \emph{software} a instalar, etc.

Valoriza-se a escrita de \emph{pouco} código que corresponda a soluções
simples e elegantes que utilizem os combinadores de ordem superior estudados
na disciplina.

\noindent \textbf{Avaliação}. Faz parte da avaliação do trabalho a sua defesa
por parte dos elementos de cada grupo. Estes devem estar preparados para
responder a perguntas sobre \emph{qualquer} dos problemas deste enunciado.
A prestação \emph{individual} de cada aluno nessa defesa oral será uma componente
importante e diferenciadora da avaliação.


\Problema

Uma serialização (ou travessia) de uma árvore é uma sua representação sob a forma de uma lista. 
Na biblioteca \ensuremath{\Conid{BTree}} encontram-se as funções de serialização \ensuremath{\Varid{inordt}}, \ensuremath{\Varid{preordt}} e \ensuremath{\Varid{postordt}},
que fazem as travessias \emph{in-order}, \emph{ pre-order} e \emph{post-order}, respectivamente.
Todas essas travessias são catamorfismos que percorrem a árvore argumento em regime \emph{depth-first}.

Pretende-se agora uma função \ensuremath{\Varid{bforder}} que faça a travessia em regime \emph{breadth-first},
isto é, por níveis.
Por exemplo, para a árvore \ensuremath{t_1 } dada em anexo e mostrada na figura a seguir,

\begin{center}
	\figura
\end{center}

\noindent a função deverá dar a lista

\begin{hscode}\SaveRestoreHook
\column{B}{@{}>{\hspre}l<{\hspost}@{}}%
\column{9}{@{}>{\hspre}l<{\hspost}@{}}%
\column{E}{@{}>{\hspre}l<{\hspost}@{}}%
\>[9]{}[\mskip1.5mu \mathrm{5},\mathrm{3},\mathrm{7},\mathrm{1},\mathrm{4},\mathrm{6},\mathrm{8}\mskip1.5mu]{}\<[E]%
\ColumnHook
\end{hscode}\resethooks

\noindent em que se vê como os níveis \ensuremath{\mathrm{5}}, depois \ensuremath{\mathrm{3},\mathrm{7}} e finalmente \ensuremath{\mathrm{1},\mathrm{4},\mathrm{6},\mathrm{8}} foram percorridos.

Pretendemos propor duas versões dessa função:

\begin{enumerate}
\item	Uma delas envolve um catamorfismo de \ensuremath{\Conid{BTree}}s:
\begin{hscode}\SaveRestoreHook
\column{B}{@{}>{\hspre}l<{\hspost}@{}}%
\column{E}{@{}>{\hspre}l<{\hspost}@{}}%
\>[B]{}\Varid{bfsLevels}\mathbin{::}\Conid{BTree}\;\Varid{a}\to [\mskip1.5mu \Varid{a}\mskip1.5mu]{}\<[E]%
\\
\>[B]{}\Varid{bfsLevels}\mathrel{=}\Varid{concat}\comp \Varid{levels}{}\<[E]%
\ColumnHook
\end{hscode}\resethooks
Complete a definição desse catamorfismo:
\begin{hscode}\SaveRestoreHook
\column{B}{@{}>{\hspre}l<{\hspost}@{}}%
\column{E}{@{}>{\hspre}l<{\hspost}@{}}%
\>[B]{}\Varid{levels}\mathbin{::}\Conid{BTree}\;\Varid{a}\to [\mskip1.5mu [\mskip1.5mu \Varid{a}\mskip1.5mu]\mskip1.5mu]{}\<[E]%
\\
\>[B]{}\Varid{levels}\mathrel{=}\llparenthesis\, \Varid{glevels}\,\rrparenthesis{}\<[E]%
\ColumnHook
\end{hscode}\resethooks
\item A segunda proposta,
\begin{hscode}\SaveRestoreHook
\column{B}{@{}>{\hspre}l<{\hspost}@{}}%
\column{E}{@{}>{\hspre}l<{\hspost}@{}}%
\>[B]{}\Varid{bft}\mathbin{::}\Conid{BTree}\;\Varid{a}\to [\mskip1.5mu \Varid{a}\mskip1.5mu]{}\<[E]%
\ColumnHook
\end{hscode}\resethooks
deverá basear-se num anamorfismo de listas.
\end{enumerate}
\textbf{Sugestão}: estudar o artigo \cite{Ok00} cujo PDF está incluído no material deste trabalho. 
Quando fizer testes ao seu código pode, se desejar, usar funções disponíveis na biblioteca
\ensuremath{\Conid{Exp}} para visualizar as árvores em GraphViz (formato .dot).

Justifique devidamente a sua resolução, que deverá vir acompanhada de diagramas explicativos.
Como já se disse, valoriza-se a escrita de \emph{pouco} código que corresponda a soluções
simples e elegantes que utilizem os combinadores de ordem superior estudados
na disciplina.



\Problema

Considere a seguinte função em Haskell:
\begin{quote}
\begin{hscode}\SaveRestoreHook
\column{B}{@{}>{\hspre}l<{\hspost}@{}}%
\column{10}{@{}>{\hspre}l<{\hspost}@{}}%
\column{11}{@{}>{\hspre}l<{\hspost}@{}}%
\column{23}{@{}>{\hspre}l<{\hspost}@{}}%
\column{32}{@{}>{\hspre}l<{\hspost}@{}}%
\column{39}{@{}>{\hspre}l<{\hspost}@{}}%
\column{46}{@{}>{\hspre}l<{\hspost}@{}}%
\column{52}{@{}>{\hspre}l<{\hspost}@{}}%
\column{E}{@{}>{\hspre}l<{\hspost}@{}}%
\>[B]{}\Varid{f}\;\Varid{x}\mathrel{=}\Varid{wrapper}\comp \Varid{worker}\;\mathbf{where}{}\<[E]%
\\
\>[B]{}\hsindent{10}{}\<[10]%
\>[10]{}\Varid{wrapper}\mathrel{=}\Varid{head}{}\<[E]%
\\
\>[B]{}\hsindent{10}{}\<[10]%
\>[10]{}\Varid{worker}\;\mathrm{0}\mathrel{=}\Varid{start}\;\Varid{x}{}\<[E]%
\\
\>[B]{}\hsindent{10}{}\<[10]%
\>[10]{}\Varid{worker}\;(\Varid{n}\mathbin{+}\mathrm{1})\mathrel{=}\Varid{loop}\;\Varid{x}\;(\Varid{worker}\;\Varid{n}){}\<[E]%
\\[\blanklineskip]%
\>[B]{}\Varid{loop}\;\Varid{x}\;{}\<[11]%
\>[11]{}[\mskip1.5mu \Varid{s},{}\<[23]%
\>[23]{}\Varid{h},{}\<[32]%
\>[32]{}\Varid{k},{}\<[39]%
\>[39]{}\Varid{j},{}\<[46]%
\>[46]{}\Varid{m}{}\<[52]%
\>[52]{}\mskip1.5mu]\mathrel{=}{}\<[E]%
\\
\>[11]{}[\mskip1.5mu \Varid{h}\mathbin{/}\Varid{k}\mathbin{+}\Varid{s},\Varid{x}\mathbin{\uparrow}\mathrm{2}\mathbin{*}\Varid{h},\Varid{k}\mathbin{*}\Varid{j},\Varid{j}\mathbin{+}\Varid{m},\Varid{m}\mathbin{+}\mathrm{8}\mskip1.5mu]{}\<[E]%
\\[\blanklineskip]%
\>[B]{}\Varid{start}\;\Varid{x}\mathrel{=}[\mskip1.5mu \Varid{x},{}\<[23]%
\>[23]{}\Varid{x}\mathbin{\uparrow}\mathrm{3},{}\<[32]%
\>[32]{}\mathrm{6},{}\<[39]%
\>[39]{}\mathrm{20},{}\<[46]%
\>[46]{}\mathrm{22}{}\<[52]%
\>[52]{}\mskip1.5mu]{}\<[E]%
\ColumnHook
\end{hscode}\resethooks
\end{quote}
Pode-se provar pela lei de recursividade mútua que \ensuremath{\Varid{f}\;\Varid{x}\;\Varid{n}} calcula o seno hiperbólico de \ensuremath{\Varid{x}},
\ensuremath{\Varid{sinh}\;\Varid{x}}, para \ensuremath{\Varid{n}} aproximações da sua série de Taylor. 
Faça a derivação da função dada a partir da referida série de Taylor, apresentando todos os cálculos justificativos, tal como se faz para outras funções no capítulo respectivo do texto base desta UC \cite{Ol98-24}.

\Problema

Quem em Braga observar, ao fim da tarde, o tráfego onde a Avenida Clairmont
Fernand se junta à N101, aproximadamente na coordenada \href{https://maps.app.goo.gl/uCbXLsdibYoochr36}{41°33'46.8"N
8°24'32.4"W} --- ver as setas da figura que se segue --- reparará nas sequências
imparáveis (infinitas!) de veículos provenientes dessas vias de circulação.

Mas também irá observar um comportamento interessante por parte dos condutores desses
veículos: por regra, \emph{cada carro numa via deixa passar, à sua frente, exactamente outro carro da outra via}. 

\begin{center}
	\mapa
\end{center}

Este comportamento \emph{civilizado} chama-se \emph{fair-merge} (ou \emph{fair-interleaving})
de duas sequências infinitas, também designadas \emph{streams} em ciência
da computação. Seja dado o tipo dessas sequências em Haskell,
\begin{hscode}\SaveRestoreHook
\column{B}{@{}>{\hspre}l<{\hspost}@{}}%
\column{E}{@{}>{\hspre}l<{\hspost}@{}}%
\>[B]{}\mathbf{data}\;\Conid{Stream}\;\Varid{a}\mathrel{=}\Conid{Cons}\;(\Varid{a},\Conid{Stream}\;\Varid{a})\;\mathbf{deriving}\;\Conid{Show}{}\<[E]%
\ColumnHook
\end{hscode}\resethooks
para o qual se define também:
\begin{hscode}\SaveRestoreHook
\column{B}{@{}>{\hspre}l<{\hspost}@{}}%
\column{E}{@{}>{\hspre}l<{\hspost}@{}}%
\>[B]{}\mathsf{out}\;(\Conid{Cons}\;(\Varid{x},\Varid{xs}))\mathrel{=}(\Varid{x},\Varid{xs}){}\<[E]%
\ColumnHook
\end{hscode}\resethooks
\noindent O referido comportamento civilizado pode definir-se, em Haskell,
da forma seguinte:\footnote{O facto das sequências serem infinitas não nos
deve preocupar, pois em Haskell isso é lidado de forma transparente por \lazy{lazy
evaluation}.}
\begin{hscode}\SaveRestoreHook
\column{B}{@{}>{\hspre}l<{\hspost}@{}}%
\column{4}{@{}>{\hspre}l<{\hspost}@{}}%
\column{E}{@{}>{\hspre}l<{\hspost}@{}}%
\>[B]{}\Varid{fair\char95 merge}\mathbin{::}(\Conid{Stream}\;\Varid{a},\Conid{Stream}\;\Varid{a})+(\Conid{Stream}\;\Varid{a},\Conid{Stream}\;\Varid{a})\to \Conid{Stream}\;\Varid{a}{}\<[E]%
\\
\>[B]{}\Varid{fair\char95 merge}\mathrel{=}\alt{\Varid{h}}{\Varid{k}}\;\mathbf{where}{}\<[E]%
\\
\>[B]{}\hsindent{4}{}\<[4]%
\>[4]{}\Varid{h}\;(\Conid{Cons}\;(\Varid{x},\Varid{xs}),\Varid{y})\mathrel{=}\Conid{Cons}\;(\Varid{x},\Varid{k}\;(\Varid{xs},\Varid{y})){}\<[E]%
\\
\>[B]{}\hsindent{4}{}\<[4]%
\>[4]{}\Varid{k}\;(\Varid{x},\Conid{Cons}\;(\Varid{y},\Varid{ys}))\mathrel{=}\Conid{Cons}\;(\Varid{y},\Varid{h}\;(\Varid{x},\Varid{ys})){}\<[E]%
\ColumnHook
\end{hscode}\resethooks

Defina \ensuremath{\Varid{fair\char95 merge}} como um \textbf{anamorfismo} de \ensuremath{\Conid{Stream}}s, usando o combinador
\begin{hscode}\SaveRestoreHook
\column{B}{@{}>{\hspre}l<{\hspost}@{}}%
\column{E}{@{}>{\hspre}l<{\hspost}@{}}%
\>[B]{}\lanabracket\,\Varid{g}\,\ranabracket\mathrel{=}\Conid{Cons}\comp (\Varid{id}\times\lanabracket\,\Varid{g}\,\ranabracket)\comp \Varid{g}{}\<[E]%
\ColumnHook
\end{hscode}\resethooks
e a seguinte estratégia:
\begin{itemize}
\item	Derivar a lei \textbf{dual} da recursividade mútua,
\begin{eqnarray}
	\ensuremath{\alt{\Varid{f}}{\Varid{g}}\mathrel{=}\ana{\alt{\Varid{h}}{\Varid{k}}}} & \equiv & \ensuremath{\begin{lcbr}\Varid{out}\comp \Varid{f}\mathrel{=}\fun F \;\alt{\Varid{f}}{\Varid{g}}\comp \Varid{h}\\\Varid{out}\comp \Varid{g}\mathrel{=}\fun F \;\alt{\Varid{f}}{\Varid{g}}\comp \Varid{k}\end{lcbr}}
	\label{eq:fokkinga_dual}
\end{eqnarray}
	tal como se fez, nas aulas, para a que está no formulário.
\item
	Usar (\ref{eq:fokkinga_dual}) na resolução do problema proposto. 
\end{itemize}
Justificar devidamente a resolução, que deverá vir acompanhada de diagramas explicativos.

\Problema

Como se sabe, é possível pensarmos em catamorfismos, anamorfismos etc \emph{probabilísticos},
quer dizer, programas recursivos que dão distribuições como resultados. Por
exemplo, podemos pensar num combinador
\begin{hscode}\SaveRestoreHook
\column{B}{@{}>{\hspre}l<{\hspost}@{}}%
\column{E}{@{}>{\hspre}l<{\hspost}@{}}%
\>[B]{}\Varid{pcataList}\mathbin{::}(()+(\Varid{a},\Varid{b})\to \fun{Dist}\;\Varid{b})\to [\mskip1.5mu \Varid{a}\mskip1.5mu]\to \fun{Dist}\;\Varid{b}{}\<[E]%
\ColumnHook
\end{hscode}\resethooks
que é muito parecido com
\begin{hscode}\SaveRestoreHook
\column{B}{@{}>{\hspre}l<{\hspost}@{}}%
\column{E}{@{}>{\hspre}l<{\hspost}@{}}%
\>[B]{}\llparenthesis\, \cdot \,\rrparenthesis\mathbin{::}(()+(\Varid{a},\Varid{b})\to \Varid{b})\to [\mskip1.5mu \Varid{a}\mskip1.5mu]\to \Varid{b}{}\<[E]%
\ColumnHook
\end{hscode}\resethooks
da biblioteca \List. A principal diferença é que o gene de \ensuremath{\Varid{pcataList}} é uma função probabilística.

Como exemplo de utilização, recorde-se que \ensuremath{\llparenthesis\, \alt{\Varid{zero}}{\Varid{add}}\,\rrparenthesis} soma todos
os elementos da lista argumento, por exemplo:
\begin{quote}
\ensuremath{\llparenthesis\, \alt{\Varid{zero}}{\Varid{add}}\,\rrparenthesis\;[\mskip1.5mu \mathrm{20},\mathrm{10},\mathrm{5}\mskip1.5mu]\mathrel{=}\mathrm{35}}.
\end{quote}
Considere-se agora a função \ensuremath{\Varid{padd}} (adição probabilística) que,
com probabilidade $90\%$ soma dois números e com probabilidade $10\%$ os subtrai:
\begin{hscode}\SaveRestoreHook
\column{B}{@{}>{\hspre}l<{\hspost}@{}}%
\column{E}{@{}>{\hspre}l<{\hspost}@{}}%
\>[B]{}\Varid{padd}\;(\Varid{a},\Varid{b})\mathrel{=}\Conid{D}\;[\mskip1.5mu (\Varid{a}\mathbin{+}\Varid{b},\mathrm{0.9}),(\Varid{a}\mathbin{-}\Varid{b},\mathrm{0.1})\mskip1.5mu]{}\<[E]%
\ColumnHook
\end{hscode}\resethooks
Se se correr
\begin{hscode}\SaveRestoreHook
\column{B}{@{}>{\hspre}l<{\hspost}@{}}%
\column{E}{@{}>{\hspre}l<{\hspost}@{}}%
\>[B]{}\Varid{d4}\mathrel{=}\Varid{pcataList}\;\alt{\Varid{pzero}}{\Varid{padd}}\;[\mskip1.5mu \mathrm{20},\mathrm{10},\mathrm{5}\mskip1.5mu]\;\mathbf{where}\;\Varid{pzero}\mathrel{=}\Varid{return}\comp \Varid{zero}{}\<[E]%
\ColumnHook
\end{hscode}\resethooks
obter-se-á:
\begin{Verbatim}[fontsize=\small]
35  81.0%
25   9.0%
 5   9.0%
15   1.0%
\end{Verbatim}

Com base neste exemplo, resolva o seguinte
\begin{quote}\em
\textbf{Problema}: Uma unidade militar pretende enviar uma mensagem urgente
a outra, mas tem o aparelho de telegrafia meio avariado. Por experiência,
o telegrafista sabe que a probabilidade de uma palavra se perder (não ser
transmitida) é $5\%$; e que, no final de cada mensagem, o aparelho envia o código
\ensuremath{\text{\ttfamily \char34 stop\char34}}, mas (por estar meio avariado), falha $10\%$ das vezes.

Qual a probabilidade de a palavra \ensuremath{\text{\ttfamily \char34 atacar\char34}} da mensagem 
\begin{quote}
\ensuremath{\Varid{words}\;\text{\ttfamily \char34 Vamos~atacar~hoje\char34}}
\end{quote}
se perder, isto é, o resultado da transmissão ser \ensuremath{[\mskip1.5mu \text{\ttfamily \char34 Vamos\char34},\text{\ttfamily \char34 hoje\char34},\text{\ttfamily \char34 stop\char34}\mskip1.5mu]}?
E a de seguirem todas as palavras, mas faltar o \ensuremath{\text{\ttfamily \char34 stop\char34}} no fim? E a da transmissão
ser perfeita?
\end{quote}

Responda a estas perguntas encontrando \ensuremath{\Varid{gene}} tal que
\begin{hscode}\SaveRestoreHook
\column{B}{@{}>{\hspre}l<{\hspost}@{}}%
\column{E}{@{}>{\hspre}l<{\hspost}@{}}%
\>[B]{}\Varid{transmitir}\mathrel{=}\Varid{pcataList}\;\Varid{gene}{}\<[E]%
\ColumnHook
\end{hscode}\resethooks
descreve o comportamento do aparelho.
Justificar devidamente a resolução, que deverá vir acompanhada de diagramas explicativos.
%

\part*{Anexos}

\appendix

\section{Natureza do trabalho a realizar}
\label{sec:documentacao}
Este trabalho teórico-prático deve ser realizado por grupos de 3 alunos.
Os detalhes da avaliação (datas para submissão do relatório e sua defesa
oral) são os que forem publicados na \cp{página da disciplina} na \emph{internet}.

Recomenda-se uma abordagem participativa dos membros do grupo em \textbf{todos}
os exercícios do trabalho, para assim poderem responder a qualquer questão
colocada na \emph{defesa oral} do relatório.

Para cumprir de forma integrada os objectivos do trabalho vamos recorrer
a uma técnica de programa\-ção dita ``\litp{literária}'' \cite{Kn92}, cujo
princípio base é o seguinte:
%
\begin{quote}\em
        Um programa e a sua documentação devem coincidir.
\end{quote}
%
Por outras palavras, o \textbf{código fonte} e a \textbf{documentação} de um
programa deverão estar no mesmo ficheiro.

O ficheiro \texttt{cp2526t.pdf} que está a ler é já um exemplo de
\litp{programação literária}: foi gerado a partir do texto fonte
\texttt{cp2526t.lhs}\footnote{O sufixo `lhs' quer dizer
\emph{\lhaskell{literate Haskell}}.} que encontrará no \MaterialPedagogico\
desta disciplina des\-com\-pactando o ficheiro \texttt{cp2526t.zip}.

Como se mostra no esquema abaixo, de um único ficheiro (\ensuremath{\Varid{lhs}})
gera-se um PDF ou faz-se a interpretação do código \Haskell\ que ele inclui:

        \esquema

Vê-se assim que, para além do \GHCi, serão necessários os executáveis \PdfLatex\ e
\LhsToTeX. Para facilitar a instalação e evitar problemas de versões e
conflitos com sistemas operativos, é recomendado o uso do \Docker\ tal como
a seguir se descreve.

\section{Docker} \label{sec:docker}

Recomenda-se o uso do \container\ cuja imagem é gerada pelo \Docker\ a partir do ficheiro
\texttt{Dockerfile} que se encontra na diretoria que resulta de descompactar
\texttt{cp2526t.zip}. Este \container\ deverá ser usado na execução
do \GHCi\ e dos comandos relativos ao \Latex. (Ver também a \texttt{Makefile}
que é disponibilizada.)

Após \href{https://docs.docker.com/engine/install/}{instalar o Docker} e
descarregar o referido zip com o código fonte do trabalho,
basta executar os seguintes comandos:
\begin{Verbatim}[fontsize=\small]
    $ docker build -t cp2526t .
    $ docker run -v ${PWD}:/cp2526t -it cp2526t
\end{Verbatim}
\textbf{NB}: O objetivo é que o container\ seja usado \emph{apenas} 
para executar o \GHCi\ e os comandos relativos ao \Latex.
Deste modo, é criado um \textit{volume} (cf.\ a opção \texttt{-v \$\{PWD\}:/cp2526t}) 
que permite que a diretoria em que se encontra na sua máquina local 
e a diretoria \texttt{/cp2526t} no \container\ sejam partilhadas.

Pretende-se então que visualize/edite os ficheiros na sua máquina local e que
os compile no \container, executando:
\begin{Verbatim}[fontsize=\small]
    $ lhs2TeX cp2526t.lhs > cp2526t.tex
    $ pdflatex cp2526t
\end{Verbatim}
\LhsToTeX\ é o pre-processador que faz ``pretty printing'' de código Haskell
em \Latex\ e que faz parte já do \container. Alternativamente, basta executar
\begin{Verbatim}[fontsize=\small]
    $ make
\end{Verbatim}
para obter o mesmo efeito que acima.

Por outro lado, o mesmo ficheiro \texttt{cp2526t.lhs} é executável e contém
o ``kit'' básico, escrito em \Haskell, para realizar o trabalho. Basta executar
\begin{Verbatim}[fontsize=\small]
    $ ghci cp2526t.lhs
\end{Verbatim}

\noindent Abra o ficheiro \texttt{cp2526t.lhs} no seu editor de texto preferido
e verifique que assim é: todo o texto que se encontra dentro do ambiente
\begin{quote}\small\tt
\text{\ttfamily \char92{}begin\char123{}code\char125{}}
\\ ... \\
\text{\ttfamily \char92{}end\char123{}code\char125{}}
\end{quote}
é seleccionado pelo \GHCi\ para ser executado.

\section{Em que consiste o TP}

Em que consiste, então, o \emph{relatório} a que se referiu acima?
É a edição do texto que está a ser lido, preenchendo o anexo \ref{sec:resolucao}
com as respostas. O relatório deverá conter ainda a identificação dos membros
do grupo de trabalho, no local respectivo da folha de rosto.

Para gerar o PDF integral do relatório deve-se ainda correr os comando seguintes,
que actualizam a bibliografia (com \Bibtex) e o índice remissivo (com \Makeindex),
\begin{Verbatim}[fontsize=\small]
    $ bibtex cp2526t.aux
    $ makeindex cp2526t.idx
\end{Verbatim}
e recompilar o texto como acima se indicou. (Como já se disse, pode fazê-lo
correndo simplesmente \texttt{make} no \container.)

No anexo \ref{sec:codigo} disponibiliza-se algum código \Haskell\ relativo
aos problemas que são colocados. Esse anexo deverá ser consultado e analisado
à medida que isso for necessário.

Deve ser feito uso da \litp{programação literária} para documentar bem o código que se
desenvolver, em particular fazendo diagramas explicativos do que foi feito e
tal como se explica no anexo \ref{sec:diagramas} que se segue.

\section{Como exprimir cálculos e diagramas em LaTeX/lhs2TeX} \label{sec:diagramas}

Como primeiro exemplo, estudar o texto fonte (\lhstotex{lhs}) do que está a ler\footnote{
Procure e.g.\ por \texttt{"sec:diagramas"}.} onde se obtém o efeito seguinte:\footnote{Exemplos
tirados de \cite{Ol98-24}.}
\begin{eqnarray*}
\start
\ensuremath{\Varid{id}\mathrel{=}\conj{\Varid{f}}{\Varid{g}}}
\just\equiv{ universal property }
\ensuremath{\begin{lcbr}\p1\comp \Varid{id}\mathrel{=}\Varid{f}\\\p2\comp \Varid{id}\mathrel{=}\Varid{g}\end{lcbr}}
\just\equiv{ identity }
\ensuremath{\begin{lcbr}\p1\mathrel{=}\Varid{f}\\\p2\mathrel{=}\Varid{g}\end{lcbr}}
\qed
\end{eqnarray*}

Os diagramas podem ser produzidos recorrendo à \emph{package} \Xymatrix, por exemplo:
\begin{eqnarray*}
\xymatrix@C=2cm{
    \ensuremath{\N_0}
           \ar[d]_-{\ensuremath{\cataNat{\Varid{g}}}}
&
    \ensuremath{\mathrm{1}\mathbin{+}\N_0}
           \ar[d]^{\ensuremath{\Varid{id}\mathbin{+}\cataNat{\Varid{g}}}}
           \ar[l]_-{\ensuremath{\mathsf{in}}}
\\
     \ensuremath{\Conid{B}}
&
     \ensuremath{\mathrm{1}\mathbin{+}\Conid{B}}
           \ar[l]^-{\ensuremath{\Varid{g}}}
}
\end{eqnarray*}

\section{O mónade das distribuições probabilísticas} \label{sec:probabilities}
Mónades são functores com propriedades adicionais que nos permitem obter
efeitos especiais em progra\-mação. Por exemplo, a biblioteca \Probability\
oferece um mónade para abordar problemas de probabilidades. Nesta biblioteca,
o conceito de distribuição estatística é captado pelo tipo
\begin{eqnarray}
     \ensuremath{\mathbf{newtype}\;\fun{Dist}\;\Varid{a}\mathrel{=}\Conid{D}\;\{\mskip1.5mu \Varid{unD}\mathbin{::}[\mskip1.5mu (\Varid{a},\Conid{ProbRep})\mskip1.5mu]\mskip1.5mu\}}
     \label{eq:Dist}
\end{eqnarray}
em que \ensuremath{\Conid{ProbRep}} é um real de \ensuremath{\mathrm{0}} a \ensuremath{\mathrm{1}}, equivalente a uma escala de $0$ a
$100 \%$.

Cada par \ensuremath{(\Varid{a},\Varid{p})} numa distribuição \ensuremath{\Varid{d}\mathbin{::}\fun{Dist}\;\Varid{a}} indica que a probabilidade
de \ensuremath{\Varid{a}} é \ensuremath{\Varid{p}}, devendo ser garantida a propriedade de  que todas as probabilidades
de \ensuremath{\Varid{d}} somam $100\%$.
Por exemplo, a seguinte distribuição de classificações por escalões de $A$ a $E$,
\[
\begin{array}{ll}
A & \rule{2mm}{3pt}\ 2\%\\
B & \rule{12mm}{3pt}\ 12\%\\
C & \rule{29mm}{3pt}\ 29\%\\
D & \rule{35mm}{3pt}\ 35\%\\
E & \rule{22mm}{3pt}\ 22\%\\
\end{array}
\]
será representada pela distribuição
\begin{hscode}\SaveRestoreHook
\column{B}{@{}>{\hspre}l<{\hspost}@{}}%
\column{E}{@{}>{\hspre}l<{\hspost}@{}}%
\>[B]{}\Varid{d1}\mathbin{::}\fun{Dist}\;\Conid{Char}{}\<[E]%
\\
\>[B]{}\Varid{d1}\mathrel{=}\Conid{D}\;[\mskip1.5mu (\text{\ttfamily 'A'},\mathrm{0.02}),(\text{\ttfamily 'B'},\mathrm{0.12}),(\text{\ttfamily 'C'},\mathrm{0.29}),(\text{\ttfamily 'D'},\mathrm{0.35}),(\text{\ttfamily 'E'},\mathrm{0.22})\mskip1.5mu]{}\<[E]%
\ColumnHook
\end{hscode}\resethooks
que o \GHCi\ mostrará assim:
\begin{Verbatim}[fontsize=\small]
'D'  35.0%
'C'  29.0%
'E'  22.0%
'B'  12.0%
'A'   2.0%
\end{Verbatim}
É possível definir geradores de distribuições, por exemplo distribuições \emph{uniformes},
\begin{hscode}\SaveRestoreHook
\column{B}{@{}>{\hspre}l<{\hspost}@{}}%
\column{E}{@{}>{\hspre}l<{\hspost}@{}}%
\>[B]{}\Varid{d2}\mathrel{=}\Varid{uniform}\;(\Varid{words}\;\text{\ttfamily \char34 Uma~frase~de~cinco~palavras\char34}){}\<[E]%
\ColumnHook
\end{hscode}\resethooks
isto é
\begin{Verbatim}[fontsize=\small]
     "Uma"  20.0%
   "cinco"  20.0%
      "de"  20.0%
   "frase"  20.0%
"palavras"  20.0%
\end{Verbatim}
distribuição \emph{normais}, eg.\
\begin{hscode}\SaveRestoreHook
\column{B}{@{}>{\hspre}l<{\hspost}@{}}%
\column{E}{@{}>{\hspre}l<{\hspost}@{}}%
\>[B]{}\Varid{d3}\mathrel{=}\Varid{normal}\;[\mskip1.5mu \mathrm{10}\mathinner{\ldotp\ldotp}\mathrm{20}\mskip1.5mu]{}\<[E]%
\ColumnHook
\end{hscode}\resethooks
etc.\footnote{Para mais detalhes ver o código fonte de \Probability, que é uma adaptação da
biblioteca \PFP\ (``Probabilistic Functional Programming''). Para quem quiser saber mais
recomenda-se a leitura do artigo \cite{EK06}.}
\ensuremath{\fun{Dist}} forma um \textbf{mónade} cuja unidade é \ensuremath{\Varid{return}\;\Varid{a}\mathrel{=}\Conid{D}\;[\mskip1.5mu (\Varid{a},\mathrm{1})\mskip1.5mu]} e cuja composição de Kleisli
é (simplificando a notação)
\begin{hscode}\SaveRestoreHook
\column{B}{@{}>{\hspre}l<{\hspost}@{}}%
\column{3}{@{}>{\hspre}l<{\hspost}@{}}%
\column{E}{@{}>{\hspre}l<{\hspost}@{}}%
\>[3]{}(\Varid{f}\kcomp \Varid{g})\;\Varid{a}\mathrel{=}[\mskip1.5mu (\Varid{y},\Varid{q}\mathbin{*}\Varid{p})\mid (\Varid{x},\Varid{p})\leftarrow \Varid{g}\;\Varid{a},(\Varid{y},\Varid{q})\leftarrow \Varid{f}\;\Varid{x}\mskip1.5mu]{}\<[E]%
\ColumnHook
\end{hscode}\resethooks
em que \ensuremath{\Varid{g}\mathbin{:}\Conid{A}\to \fun{Dist}\;\mathit B} e \ensuremath{\Varid{f}\mathbin{:}\mathit B\to \fun{Dist}\;\mathit C} são funções \textbf{monádicas} que representam
\emph{computações probabilísticas}.

Este mónade é adequado à resolução de problemas de \emph{probabilidades e estatística} usando programação funcional, de forma elegante e como caso particular da programação monádica.

\section{Código fornecido}\label{sec:codigo}

\subsection*{Problema 1}

Árvores exemplo:
\begin{hscode}\SaveRestoreHook
\column{B}{@{}>{\hspre}l<{\hspost}@{}}%
\column{3}{@{}>{\hspre}l<{\hspost}@{}}%
\column{4}{@{}>{\hspre}l<{\hspost}@{}}%
\column{5}{@{}>{\hspre}l<{\hspost}@{}}%
\column{12}{@{}>{\hspre}l<{\hspost}@{}}%
\column{E}{@{}>{\hspre}l<{\hspost}@{}}%
\>[B]{}t_1 \mathbin{::}\Conid{BTree}\;\Conid{Int}{}\<[E]%
\\
\>[B]{}t_1 \mathrel{=}\Conid{Node}\;(\mathrm{5},(\Conid{Node}\;(\mathrm{3},(\Conid{Node}\;(\mathrm{1},(\Conid{Empty},\Conid{Empty})),\Conid{Node}\;(\mathrm{4},(\Conid{Empty},\Conid{Empty})))),{}\<[E]%
\\
\>[B]{}\hsindent{12}{}\<[12]%
\>[12]{}\Conid{Node}\;(\mathrm{7},(\Conid{Node}\;(\mathrm{6},(\Conid{Empty},\Conid{Empty})),\Conid{Node}\;(\mathrm{8},(\Conid{Empty},\Conid{Empty})))))){}\<[E]%
\\[\blanklineskip]%
\>[B]{}t_2 \mathbin{::}\Conid{BTree}\;\Conid{Int}{}\<[E]%
\\
\>[B]{}t_2 \mathrel{=}{}\<[E]%
\\
\>[B]{}\hsindent{3}{}\<[3]%
\>[3]{}\Varid{node}\;\mathrm{1}\;{}\<[E]%
\\
\>[3]{}\hsindent{2}{}\<[5]%
\>[5]{}(\Varid{node}\;\mathrm{2}\;(\Varid{node}\;\mathrm{4}\;\Conid{Empty}\;\Conid{Empty})\;(\Varid{node}\;\mathrm{5}\;\Conid{Empty}\;\Conid{Empty}))\;{}\<[E]%
\\
\>[3]{}\hsindent{2}{}\<[5]%
\>[5]{}(\Varid{node}\;\mathrm{3}\;(\Varid{node}\;\mathrm{6}\;\Conid{Empty}\;\Conid{Empty})\;(\Varid{node}\;\mathrm{7}\;\Conid{Empty}\;\Conid{Empty})){}\<[E]%
\\[\blanklineskip]%
\>[B]{}t_3 \mathbin{::}\Conid{BTree}\;\Conid{Char}{}\<[E]%
\\
\>[B]{}t_3 \mathrel{=}{}\<[E]%
\\
\>[B]{}\hsindent{3}{}\<[3]%
\>[3]{}\Varid{node}\;\text{\ttfamily 'A'}\;{}\<[E]%
\\
\>[3]{}\hsindent{2}{}\<[5]%
\>[5]{}(\Varid{node}\;\text{\ttfamily 'B'}\;(\Varid{node}\;\text{\ttfamily 'C'}\;(\Varid{node}\;\text{\ttfamily 'D'}\;\Conid{Empty}\;\Conid{Empty})\;\Conid{Empty})\;\Conid{Empty})\;{}\<[E]%
\\
\>[3]{}\hsindent{2}{}\<[5]%
\>[5]{}(\Varid{node}\;\text{\ttfamily 'E'}\;\Conid{Empty}\;\Conid{Empty}){}\<[E]%
\\[\blanklineskip]%
\>[B]{}t_4 \mathbin{::}\Conid{BTree}\;\Conid{Char}{}\<[E]%
\\
\>[B]{}t_4 \mathrel{=}{}\<[E]%
\\
\>[B]{}\hsindent{3}{}\<[3]%
\>[3]{}\Varid{node}\;\text{\ttfamily 'A'}\;{}\<[E]%
\\
\>[3]{}\hsindent{2}{}\<[5]%
\>[5]{}(\Varid{node}\;\text{\ttfamily 'B'}\;(\Varid{node}\;\text{\ttfamily 'C'}\;(\Varid{node}\;\text{\ttfamily 'D'}\;\Conid{Empty}\;\Conid{Empty})\;\Conid{Empty})\;\Conid{Empty})\;{}\<[E]%
\\
\>[3]{}\hsindent{2}{}\<[5]%
\>[5]{}\Conid{Empty}{}\<[E]%
\\[\blanklineskip]%
\>[B]{}t_5 \mathbin{::}\Conid{BTree}\;\Conid{Int}{}\<[E]%
\\
\>[B]{}t_5 \mathrel{=}{}\<[E]%
\\
\>[B]{}\hsindent{3}{}\<[3]%
\>[3]{}\Varid{node}\;\mathrm{1}\;{}\<[E]%
\\
\>[3]{}\hsindent{1}{}\<[4]%
\>[4]{}(\Varid{node}\;\mathrm{2}\;(\Varid{node}\;\mathrm{4}\;\Conid{Empty}\;\Conid{Empty})\;\Conid{Empty})\;{}\<[E]%
\\
\>[3]{}\hsindent{1}{}\<[4]%
\>[4]{}(\Varid{node}\;\mathrm{3}\;\Conid{Empty}\;(\Varid{node}\;\mathrm{5}\;(\Varid{node}\;\mathrm{6}\;\Conid{Empty}\;\Conid{Empty})\;\Conid{Empty})){}\<[E]%
\\[\blanklineskip]%
\>[B]{}\Varid{node}\;\Varid{a}\;\Varid{b}\;\Varid{c}\mathrel{=}\Conid{Node}\;(\Varid{a},(\Varid{b},\Varid{c})){}\<[E]%
\ColumnHook
\end{hscode}\resethooks

%----------------- Soluções dos alunos -----------------------------------------%

\section{Soluções dos alunos}\label{sec:resolucao}
Os alunos devem colocar neste anexo as suas soluções para os exercícios
propostos, de acordo com o ``layout'' que se fornece.
Não podem ser alterados os nomes ou tipos das funções dadas, mas pode ser
adicionado texto ao anexo, bem como diagramas e/ou outras funções auxiliares
que sejam necessárias.

\noindent
\textbf{Importante}: Não pode ser alterado o texto deste ficheiro fora deste anexo.

\subsection*{Problema 1}

Na \textbf{primeira questão} do Problema 1, é-nos pedido uma implementação da função 
\ensuremath{\Varid{levels}} que coloca os elementos de uma árvore binária numa lista, tendo 
esta várias listas, cada uma relativa a um nível da árvore.
\begin{hscode}\SaveRestoreHook
\column{B}{@{}>{\hspre}l<{\hspost}@{}}%
\column{5}{@{}>{\hspre}l<{\hspost}@{}}%
\column{9}{@{}>{\hspre}l<{\hspost}@{}}%
\column{E}{@{}>{\hspre}l<{\hspost}@{}}%
\>[B]{}\Varid{glevels}\mathrel{=}\alt{\Varid{nil}}{\Varid{cons}\comp (\Varid{singl}\times\Varid{g})}{}\<[E]%
\\
\>[B]{}\hsindent{5}{}\<[5]%
\>[5]{}\mathbf{where}{}\<[E]%
\\
\>[5]{}\hsindent{4}{}\<[9]%
\>[9]{}\Varid{g}\;([\mskip1.5mu \mskip1.5mu],\Varid{r})\mathrel{=}\Varid{r}{}\<[E]%
\\
\>[5]{}\hsindent{4}{}\<[9]%
\>[9]{}\Varid{g}\;(\Varid{l},[\mskip1.5mu \mskip1.5mu])\mathrel{=}\Varid{l}{}\<[E]%
\\
\>[5]{}\hsindent{4}{}\<[9]%
\>[9]{}\Varid{g}\;((\Varid{l}\mathbin{:}\Varid{ls}),(\Varid{r}\mathbin{:}\Varid{rs}))\mathrel{=}(\Varid{l}\mathbin{+\!\!+}\Varid{r})\mathbin{:}\Varid{g}\;(\Varid{ls},\Varid{rs}){}\<[E]%
\ColumnHook
\end{hscode}\resethooks

O diagrama do catamorfismo \ensuremath{\Varid{levels}} é o seguinte:
\begin{eqnarray*}
\xymatrix@C=3cm@R=2cm{
        \ensuremath{\Conid{BTree}\;\Conid{A}}
                \ar[d]_-{\ensuremath{\llparenthesis\, \Varid{glevels}\,\rrparenthesis}}
                \ar@/^1pc/[r]^-{\ensuremath{\Varid{outBTree}}}
&
        \ensuremath{\mathrm{1}\mathbin{+}\Conid{A}\times(\Conid{BTree}\;\Conid{A})^2}
                \ar[d]^{\ensuremath{\Varid{id}\mathbin{+}\Varid{id}\times\llparenthesis\, \Varid{glevels}\,\rrparenthesis^2}}
                \ar@/^1pc/[l]^-{\ensuremath{\Varid{inBTree}}}
\\
        \ensuremath{{{\Conid{A}}^{*}}^{*}}
&
        \ensuremath{\mathrm{1}\mathbin{+}\Conid{A}\times({{\Conid{A}}^{*}}^{*})^2}
                \ar[l]^-{\ensuremath{\Varid{glevels}}}
}
\end{eqnarray*}

O gene \ensuremath{\Varid{glevels}} deverá ser capaz de juntar os níveis de duas sub-árvores 
e adicionar a raiz da àrvore à cabeça da lista resultante. Logo será necessário
realizar uma espécie de \ensuremath{\Varid{zip}} aos elementos das listas. Esta operação está pré definida em Haskell, 
no entanto esta função remove elementos de uma das listas, caso não tenham o mesmo comprimento, 
então implementamos a função auxiliar \ensuremath{\Varid{g}} que concatena os elementos (listas) de 
duas listas, sem perder informação.

O diagrama do gene \ensuremath{\Varid{glevels}} é o seguinte:
\begin{eqnarray*}
\xymatrix@C=2cm@R=2cm{
        \ensuremath{\mathrm{1}}
                \ar[r]^-{\ensuremath{i_1}}
                \ar[d]_-{\ensuremath{\Varid{nil}}}
&
        \ensuremath{\mathrm{1}\mathbin{+}\Conid{A}\times({{\Conid{A}}^{*}}^{*})^2}
                \ar[d]_{\ensuremath{\Varid{glevels}}}
&
        \ensuremath{\Conid{A}\times({{\Conid{A}}^{*}}^{*})^2}
                \ar[l]_-{\ensuremath{i_2}}
                \ar[d]^-{\ensuremath{\Varid{singl}\times\Varid{g}}}
\\
        \ensuremath{{{\Conid{A}}^{*}}^{*}}
                \ar[r]_{\ensuremath{\Varid{id}}}
&
        \ensuremath{{{\Conid{A}}^{*}}^{*}}
&
        \ensuremath{{\Conid{A}}^{*}\times{{\Conid{A}}^{*}}^{*}}
                \ar[l]^{\ensuremath{\Varid{cons}}}
}
\end{eqnarray*}

Na \textbf{segunda questão} deste problema, o desafio baseia-se em implementar uma travessia
em largura através de um anamorfismo de listas. A estratégia usada passa por usar uma
lista de árvores, que corresponde à queue de nodos da árvore a explorar.
\begin{hscode}\SaveRestoreHook
\column{B}{@{}>{\hspre}l<{\hspost}@{}}%
\column{5}{@{}>{\hspre}l<{\hspost}@{}}%
\column{9}{@{}>{\hspre}l<{\hspost}@{}}%
\column{E}{@{}>{\hspre}l<{\hspost}@{}}%
\>[B]{}\Varid{bft}\mathrel{=}\Varid{concat}\comp \anaList{(\Varid{id}+\Varid{g})\comp \Varid{outList}}\comp \Varid{singl}{}\<[E]%
\\
\>[B]{}\hsindent{5}{}\<[5]%
\>[5]{}\mathbf{where}{}\<[E]%
\\
\>[5]{}\hsindent{4}{}\<[9]%
\>[9]{}\Varid{g}\mathbin{::}(\Conid{BTree}\;\Varid{a},[\mskip1.5mu \Conid{BTree}\;\Varid{a}\mskip1.5mu])\to ([\mskip1.5mu \Varid{a}\mskip1.5mu],[\mskip1.5mu \Conid{BTree}\;\Varid{a}\mskip1.5mu]){}\<[E]%
\\
\>[5]{}\hsindent{4}{}\<[9]%
\>[9]{}\Varid{g}\;(\Conid{Empty},\Varid{rest})\mathrel{=}([\mskip1.5mu \mskip1.5mu],\Varid{rest}){}\<[E]%
\\
\>[5]{}\hsindent{4}{}\<[9]%
\>[9]{}\Varid{g}\;(\Conid{Node}\;(\Varid{x},(\Varid{l},\Varid{r})),\Varid{rest})\mathrel{=}([\mskip1.5mu \Varid{x}\mskip1.5mu],\Varid{rest}\mathbin{+\!\!+}[\mskip1.5mu \Varid{l},\Varid{r}\mskip1.5mu]){}\<[E]%
\ColumnHook
\end{hscode}\resethooks

O diagrama deste anamorfismo é o seguinte:
\begin{eqnarray*}
\xymatrix@C=4em@R=4em{
        \ensuremath{{{\Conid{A}}^{*}}^{*}}
                \ar@/^1pc/[rr]^-{\ensuremath{\Varid{outList}}}
&
&
        \ensuremath{\mathrm{1}\mathbin{+}{\Conid{A}}^{*}\times{{\Conid{A}}^{*}}^{*}}
                \ar@/^1pc/[ll]^-{\ensuremath{\Varid{inList}}}
\\
        \ensuremath{{(\Conid{BTree}\;\Conid{A})}^{*}}
                \ar[u]^-{\ensuremath{\lanabracket\,(\Varid{id}\mathbin{+}\Varid{g})\comp \Varid{out}\,\ranabracket}}
                \ar[r]_-{\ensuremath{\Varid{outList}}}
&
        \ensuremath{\mathrm{1}\mathbin{+}\Conid{BTree}\;\Conid{A}\times{(\Conid{BTree}\;\Conid{A})}^{*}}
                \ar[r]_-{\ensuremath{\Varid{id}\mathbin{+}\Varid{g}}}
&
        \ensuremath{\mathrm{1}\mathbin{+}{\Conid{A}}^{*}\times{(\Conid{BTree}\;\Conid{A})}^{*}}
                \ar[u]_{\ensuremath{\Varid{id}\mathbin{+}\Varid{id}\times\lanabracket\,(\Varid{id}\mathbin{+}\Varid{g})\comp \Varid{out}\,\ranabracket}}
}
\end{eqnarray*}

\subsection*{Problema 2}

Neste problema é-nos proposto derivar a definição do seno hiperbólico para n aproximações da sua série de Taylor.
\begin{eqnarray*}
\start
\ensuremath{\Varid{sinh}\;(\Varid{x})\mathrel{=}{\sum}\;\frac{\Varid{x}\mathbin{\uparrow}(\mathrm{2}\;\Varid{n}\mathbin{+}\mathrm{1})}{{(\mathrm{2}\;\Varid{n}\mathbin{+}\mathrm{1})!}}}
\end{eqnarray*}

O primeiro passo será resolver o somatório, transformando a definição de \ensuremath{\Varid{sinh}\;\Varid{x}} numa função recursiva:
\begin{hscode}\SaveRestoreHook
\column{B}{@{}>{\hspre}l<{\hspost}@{}}%
\column{E}{@{}>{\hspre}l<{\hspost}@{}}%
\>[B]{}\Varid{sinh}\;\Varid{x}\;\mathrm{0}\mathrel{=}\Varid{x}{}\<[E]%
\\
\>[B]{}\Varid{sinh}\;\Varid{x}\;(\Varid{n}\mathbin{+}\mathrm{1})\mathrel{=}\Varid{x}\mathbin{\uparrow}(\mathrm{2}\mathbin{*}\Varid{n}\mathbin{+}\mathrm{3})\mathbin{/}{(\mathrm{2}\mathbin{*}\Varid{n}\mathbin{+}\mathrm{3})!}\mathbin{+}\Varid{sinh}\;\Varid{x}\;\Varid{n}{}\<[E]%
\ColumnHook
\end{hscode}\resethooks

O corpo desta função é dependente do seu input \ensuremath{\Varid{n}\mathbin{+}\mathrm{1}}, logo teremos que simplificar \ensuremath{\Varid{sinh}\;\Varid{x}}. Podemos então introduzir duas funções:
\begin{hscode}\SaveRestoreHook
\column{B}{@{}>{\hspre}l<{\hspost}@{}}%
\column{E}{@{}>{\hspre}l<{\hspost}@{}}%
\>[B]{}\Varid{f1}\;\Varid{x}\;\Varid{n}\mathrel{=}\Varid{x}\mathbin{\uparrow}(\mathrm{2}\mathbin{*}\Varid{n}\mathbin{+}\mathrm{3}){}\<[E]%
\\[\blanklineskip]%
\>[B]{}\Varid{f2}\;\Varid{n}\mathrel{=}{(\mathrm{2}\mathbin{*}\Varid{n}\mathbin{+}\mathrm{3})!}{}\<[E]%
\ColumnHook
\end{hscode}\resethooks

Desta forma \ensuremath{\Varid{sinh}\;\Varid{x}} ficará dependente das definições de \ensuremath{\Varid{f1}} e \ensuremath{\Varid{f2}}. Será necessário obter as definições recursivas primitivas destas duas funções, tal é alcançável através da substituição do valor de \ensuremath{\Varid{n}} por 0, de forma a obter a base da recursividade:
\begin{hscode}\SaveRestoreHook
\column{B}{@{}>{\hspre}l<{\hspost}@{}}%
\column{E}{@{}>{\hspre}l<{\hspost}@{}}%
\>[B]{}\Varid{f1}\;\Varid{x}\;\mathrm{0}\mathrel{=}\Varid{x}\mathbin{\uparrow}(\mathrm{2}\mathbin{*}\mathrm{0}\mathbin{+}\mathrm{3})\mathrel{=}\Varid{x}\mathbin{\uparrow}\mathrm{3}{}\<[E]%
\\[\blanklineskip]%
\>[B]{}\Varid{f2}\;\mathrm{0}\mathrel{=}{(\mathrm{2}\mathbin{*}\mathrm{0}\mathbin{+}\mathrm{3})!}\mathrel{=}\mathrm{6}{}\<[E]%
\ColumnHook
\end{hscode}\resethooks

A definição recursiva de \ensuremath{\Varid{f1}} pode ser obtida através da divisão entre \ensuremath{\Varid{f1}\;\Varid{x}\;(\Varid{n}\mathbin{+}\mathrm{1})} e \ensuremath{\Varid{f1}\;\Varid{x}\;\Varid{n}}:
\begin{eqnarray*}
\start
\ensuremath{\frac{\Varid{f1}\;\Varid{x}\;(\Varid{n}\mathbin{+}\mathrm{1})}{\Varid{f1}\;\Varid{x}\;\Varid{n}}\mathrel{=}\frac{\Varid{x}\mathbin{\uparrow}(\mathrm{2}\;\Varid{n}\mathbin{+}\mathrm{5})}{\Varid{x}\mathbin{\uparrow}(\mathrm{2}\;\Varid{n}\mathbin{+}\mathrm{3})}\mathrel{=}\Varid{x}\mathbin{\uparrow}\mathrm{2}}
\end{eqnarray*}

Podemos aplicar o mesmo procedimento à função \ensuremath{\Varid{f2}}:
\begin{eqnarray*}
\start
\ensuremath{\frac{\Varid{f2}\;\Varid{x}\;(\Varid{n}\mathbin{+}\mathrm{1})}{\Varid{f2}\;\Varid{x}\;\Varid{n}}\mathrel{=}\frac{{(\mathrm{2}\;\Varid{n}\mathbin{+}\mathrm{5})!}}{{(\mathrm{2}\;\Varid{n}\mathbin{+}\mathrm{3})!}}\mathrel{=}(\mathrm{2}\;\Varid{n}\mathbin{+}\mathrm{5})\mathbin{*}(\mathrm{2}\;\Varid{n}\mathbin{+}\mathrm{4})\mathrel{=}\mathrm{4}\;\Varid{n}\mathbin{\uparrow}\mathrm{2}\mathbin{+}\mathrm{18}\;\Varid{n}\mathbin{+}\mathrm{20}}
\end{eqnarray*}

Temos então as definições recursivas de \ensuremath{\Varid{f1}} e \ensuremath{\Varid{f2}}:
\begin{hscode}\SaveRestoreHook
\column{B}{@{}>{\hspre}l<{\hspost}@{}}%
\column{E}{@{}>{\hspre}l<{\hspost}@{}}%
\>[B]{}\Varid{f1}\;\Varid{x}\;\mathrm{0}\mathrel{=}\Varid{x}\mathbin{\uparrow}\mathrm{3}{}\<[E]%
\\
\>[B]{}\Varid{f1}\;\Varid{x}\;(\Varid{n}\mathbin{+}\mathrm{1})\mathrel{=}\Varid{x}\mathbin{\uparrow}\mathrm{2}\mathbin{*}\Varid{f1}\;\Varid{x}\;\Varid{n}{}\<[E]%
\\[\blanklineskip]%
\>[B]{}\Varid{f2}\;\mathrm{0}\mathrel{=}\mathrm{6}{}\<[E]%
\\
\>[B]{}\Varid{f2}\;(\Varid{n}\mathbin{+}\mathrm{1})\mathrel{=}(\mathrm{4}\mathbin{*}\Varid{n}\mathbin{\uparrow}\mathrm{2}\mathbin{+}\mathrm{18}\mathbin{*}\Varid{n}\mathbin{+}\mathrm{20})\mathbin{*}\Varid{f2}\;\Varid{n}{}\<[E]%
\ColumnHook
\end{hscode}\resethooks

Podemos verificar que \ensuremath{\Varid{f2}} ainda depende de \ensuremath{\Varid{n}}, logo teremos de repetir o processo de forma a simplificar \ensuremath{\Varid{f2}}. Introduzimos \ensuremath{\Varid{f3}\mathrel{=}\mathrm{4}\mathbin{*}\Varid{n}\mathbin{\uparrow}\mathrm{2}\mathbin{+}\mathrm{18}\mathbin{*}\Varid{n}\mathbin{+}\mathrm{20}}. Por substituição obtemos o caso base de \ensuremath{\Varid{f3}} e resolvendo a equação \ensuremath{\Varid{f3}\;(\Varid{n}\mathbin{+}\mathrm{1})\mathbin{-}\Varid{f3}\;\Varid{n}} obtemos a definição recursiva:
\begin{hscode}\SaveRestoreHook
\column{B}{@{}>{\hspre}l<{\hspost}@{}}%
\column{E}{@{}>{\hspre}l<{\hspost}@{}}%
\>[B]{}\Varid{f3}\;\mathrm{0}\mathrel{=}\mathrm{20}{}\<[E]%
\\
\>[B]{}\Varid{f3}\;(\Varid{n}\mathbin{+}\mathrm{1})\mathrel{=}\mathrm{8}\mathbin{*}\Varid{n}\mathbin{+}\mathrm{22}\mathbin{+}\Varid{f3}\;\Varid{n}{}\<[E]%
\ColumnHook
\end{hscode}\resethooks

Será necessário repetir o processo uma última vez, pois \ensuremath{\Varid{f3}} continua dependente de \ensuremath{\Varid{n}}. Para obter \ensuremath{\Varid{f4}} usamos o mesmo processo apresentado anteriormente, isto é, tomamos \ensuremath{\Varid{f4}\;\Varid{n}\mathrel{=}\mathrm{8}\mathbin{*}\Varid{n}}, substituimos \ensuremath{\Varid{n}} por 0, para determinar a base da recursividade, e calculamos \ensuremath{\Varid{f4}\;(\Varid{n}\mathbin{+}\mathrm{1})\mathbin{-}\Varid{f4}\;\Varid{n}} para determinar o corpo da recursividade:
\begin{hscode}\SaveRestoreHook
\column{B}{@{}>{\hspre}l<{\hspost}@{}}%
\column{E}{@{}>{\hspre}l<{\hspost}@{}}%
\>[B]{}\Varid{f4}\;\mathrm{0}\mathrel{=}\mathrm{22}{}\<[E]%
\\
\>[B]{}\Varid{f4}\;(\Varid{n}\mathbin{+}\mathrm{1})\mathrel{=}\mathrm{8}\mathbin{+}\Varid{f4}\;\Varid{n}{}\<[E]%
\ColumnHook
\end{hscode}\resethooks

Podemos então apresentar as versões primitivas das funções mencionadas anteriormente, que nos irão ajudar na dedução da função \ensuremath{\Varid{f}}:
\begin{hscode}\SaveRestoreHook
\column{B}{@{}>{\hspre}l<{\hspost}@{}}%
\column{E}{@{}>{\hspre}l<{\hspost}@{}}%
\>[B]{}\Varid{f4}\;\mathrm{0}\mathrel{=}\mathrm{22}{}\<[E]%
\\
\>[B]{}\Varid{f4}\;(\Varid{n}\mathbin{+}\mathrm{1})\mathrel{=}\mathrm{8}\mathbin{+}\Varid{f4}\;\Varid{n}{}\<[E]%
\\[\blanklineskip]%
\>[B]{}\Varid{f3'}\;\mathrm{0}\mathrel{=}\mathrm{20}{}\<[E]%
\\
\>[B]{}\Varid{f3'}\;(\Varid{n}\mathbin{+}\mathrm{1})\mathrel{=}\Varid{f4}\;\Varid{n}\mathbin{+}\Varid{f3'}\;\Varid{n}{}\<[E]%
\\[\blanklineskip]%
\>[B]{}\Varid{f2'}\;\mathrm{0}\mathrel{=}\mathrm{6}{}\<[E]%
\\
\>[B]{}\Varid{f2'}\;(\Varid{n}\mathbin{+}\mathrm{1})\mathrel{=}\Varid{f3'}\;\Varid{n}\mathbin{*}\Varid{f2'}\;\Varid{n}{}\<[E]%
\\[\blanklineskip]%
\>[B]{}\Varid{f1}\;\Varid{x}\;\mathrm{0}\mathrel{=}\Varid{x}\mathbin{\uparrow}\mathrm{3}{}\<[E]%
\\
\>[B]{}\Varid{f1}\;\Varid{x}\;(\Varid{n}\mathbin{+}\mathrm{1})\mathrel{=}\Varid{x}\mathbin{\uparrow}\mathrm{2}\mathbin{*}\Varid{f1}\;\Varid{x}\;\Varid{n}{}\<[E]%
\\[\blanklineskip]%
\>[B]{}\Varid{sinh'}\;\Varid{x}\;\mathrm{0}\mathrel{=}\Varid{x}{}\<[E]%
\\
\>[B]{}\Varid{sinh'}\;\Varid{x}\;(\Varid{n}\mathbin{+}\mathrm{1})\mathrel{=}\Varid{f1}\;\Varid{x}\;\Varid{n}\mathbin{/}\Varid{f2'}\;\Varid{n}\mathbin{+}\Varid{sinh'}\;\Varid{x}\;\Varid{n}{}\<[E]%
\ColumnHook
\end{hscode}\resethooks

Todas estas funções partilham uma caracteristica fundamental para que seja possível derivar \ensuremath{\Varid{f}}, todas elas trabalham sobre o Functor dos Naturais. Através das seguintes regras, podemos derivar uma versão de \ensuremath{\Varid{f}}, que usa o catamorfismo \ensuremath{\for{\Varid{b}}\ {\Varid{i}}} dos Naturais:
\begin{itemize}
\item O corpo do ciclo loop terá tantos argumentos quanto o número de funções mutuamente recursivas.
\item Para as variáveis escolhem-se os próprios nomes das funções, pela ordem que se achar conveniente.
\item Para os resultados vão-se buscar as expressões respectivas, retirando a variável n.
\item Em start coleccionam-se os resultados dos casos de base das funções, pela mesma ordem.
\end{itemize}

Seguindo as regras apresentadas e a \textbf{lei da recursividade mútua}, podemos obter a seguinte função (\textbf{Nota}: por simplicidade e elegância, em vez de se utilizar pares de elementos, usaremos uma lista e a função head):
\begin{hscode}\SaveRestoreHook
\column{B}{@{}>{\hspre}l<{\hspost}@{}}%
\column{9}{@{}>{\hspre}l<{\hspost}@{}}%
\column{14}{@{}>{\hspre}l<{\hspost}@{}}%
\column{E}{@{}>{\hspre}l<{\hspost}@{}}%
\>[B]{}\Varid{f'}\;\Varid{x}\mathrel{=}\Varid{head}\comp \for{\Varid{loop}}\ {\Varid{start}}\;\mathbf{where}{}\<[E]%
\\
\>[B]{}\hsindent{9}{}\<[9]%
\>[9]{}\Varid{loop}\;[\mskip1.5mu \Varid{s},\Varid{f1},\Varid{f2},\Varid{f3},\Varid{f4}\mskip1.5mu]\mathrel{=}{}\<[E]%
\\
\>[9]{}\hsindent{5}{}\<[14]%
\>[14]{}[\mskip1.5mu \Varid{f1}\mathbin{/}\Varid{f2}\mathbin{+}\Varid{s},\Varid{x}\mathbin{\uparrow}\mathrm{2}\mathbin{*}\Varid{f1},\Varid{f2}\mathbin{*}\Varid{f3},\Varid{f3}\mathbin{+}\Varid{f4},\mathrm{8}\mathbin{+}\Varid{f4}\mskip1.5mu]{}\<[E]%
\\
\>[B]{}\hsindent{9}{}\<[9]%
\>[9]{}\Varid{start}\;\Varid{x}\mathrel{=}[\mskip1.5mu \Varid{x},\Varid{x}\mathbin{\uparrow}\mathrm{3},\mathrm{6},\mathrm{20},\mathrm{22}\mskip1.5mu]{}\<[E]%
\ColumnHook
\end{hscode}\resethooks

No entanto, a função \ensuremath{\Varid{f}} está definida em formato pointwise, logo será necessário aplicar a regra Universal-Cata a \ensuremath{\for{\Varid{loop}}\ {\Varid{start}}}:
\begin{eqnarray*}
\start
\ensuremath{\Varid{worker}\mathrel{=}\for{\Varid{loop}}\ {\Varid{start}}}
\just\equiv{ Def. for, universal cata }
\ensuremath{\Varid{worker}\comp \mathsf{in}\mathrel{=}\alt{\Varid{start}}{\Varid{loop}}\comp \fun F \;\Varid{worker}}
\just\equiv{ \ensuremath{\fun F \;\Varid{f}\mathrel{=}\Varid{id}\mathbin{+}\Varid{f}}, Absorção-\ensuremath{\mathbin{+}}, Natural-\ensuremath{\Varid{id}} }
\ensuremath{\Varid{worker}\comp \mathsf{in}\mathrel{=}\alt{\Varid{start}}{\Varid{loop}\comp \Varid{worker}}}
\just\equiv{ Def. \ensuremath{\mathsf{in}}, Fusão-\ensuremath{\mathbin{+}}, Eq-\ensuremath{\mathbin{+}}, pointwise, Def. composição }
\ensuremath{\begin{lcbr}\Varid{worker}\;\mathrm{0}\mathrel{=}\Varid{start}\;\Varid{x}\\\Varid{worker}\;(\Varid{n}\mathbin{+}\mathrm{1})\mathrel{=}\Varid{loop}\;(\Varid{worker}\;\Varid{n})\end{lcbr}}
\end{eqnarray*}

Através da definição pointwise de \ensuremath{\Varid{worker}}, e de algumas mudanças de nome às variáveis, derivamos a definição do seno hiperbólico de \ensuremath{\Varid{x}}, para \ensuremath{\Varid{n}} aproximações da sua série de Taylor:
\begin{hscode}\SaveRestoreHook
\column{B}{@{}>{\hspre}l<{\hspost}@{}}%
\column{10}{@{}>{\hspre}l<{\hspost}@{}}%
\column{11}{@{}>{\hspre}l<{\hspost}@{}}%
\column{23}{@{}>{\hspre}l<{\hspost}@{}}%
\column{32}{@{}>{\hspre}l<{\hspost}@{}}%
\column{39}{@{}>{\hspre}l<{\hspost}@{}}%
\column{46}{@{}>{\hspre}l<{\hspost}@{}}%
\column{52}{@{}>{\hspre}l<{\hspost}@{}}%
\column{E}{@{}>{\hspre}l<{\hspost}@{}}%
\>[B]{}\Varid{f}\;\Varid{x}\mathrel{=}\Varid{wrapper}\comp \Varid{worker}\;\mathbf{where}{}\<[E]%
\\
\>[B]{}\hsindent{10}{}\<[10]%
\>[10]{}\Varid{wrapper}\mathrel{=}\Varid{head}{}\<[E]%
\\
\>[B]{}\hsindent{10}{}\<[10]%
\>[10]{}\Varid{worker}\;\mathrm{0}\mathrel{=}\Varid{start}\;\Varid{x}{}\<[E]%
\\
\>[B]{}\hsindent{10}{}\<[10]%
\>[10]{}\Varid{worker}\;(\Varid{n}\mathbin{+}\mathrm{1})\mathrel{=}\Varid{loop}\;\Varid{x}\;(\Varid{worker}\;\Varid{n}){}\<[E]%
\\[\blanklineskip]%
\>[B]{}\Varid{loop}\;\Varid{x}\;{}\<[11]%
\>[11]{}[\mskip1.5mu \Varid{s},{}\<[23]%
\>[23]{}\Varid{h},{}\<[32]%
\>[32]{}\Varid{k},{}\<[39]%
\>[39]{}\Varid{j},{}\<[46]%
\>[46]{}\Varid{m}{}\<[52]%
\>[52]{}\mskip1.5mu]\mathrel{=}{}\<[E]%
\\
\>[11]{}[\mskip1.5mu \Varid{h}\mathbin{/}\Varid{k}\mathbin{+}\Varid{s},\Varid{x}\mathbin{\uparrow}\mathrm{2}\mathbin{*}\Varid{h},\Varid{k}\mathbin{*}\Varid{j},\Varid{j}\mathbin{+}\Varid{m},\Varid{m}\mathbin{+}\mathrm{8}\mskip1.5mu]{}\<[E]%
\\[\blanklineskip]%
\>[B]{}\Varid{start}\;\Varid{x}\mathrel{=}[\mskip1.5mu \Varid{x},{}\<[23]%
\>[23]{}\Varid{x}\mathbin{\uparrow}\mathrm{3},{}\<[32]%
\>[32]{}\mathrm{6},{}\<[39]%
\>[39]{}\mathrm{20},{}\<[46]%
\>[46]{}\mathrm{22}{}\<[52]%
\>[52]{}\mskip1.5mu]{}\<[E]%
\ColumnHook
\end{hscode}\resethooks


\subsection*{Problema 3}

O primeiro passo para resolver este problema será derivar a lei \textbf{dual} da recursividade mútua:
\begin{eqnarray*}
\start
\ensuremath{\alt{\Varid{f}}{\Varid{g}}\mathrel{=}\ana{\alt{\Varid{h}}{\Varid{k}}}}
\just\equiv{ universal ana }
\ensuremath{\Varid{out}\comp \alt{\Varid{f}}{\Varid{g}}\mathrel{=}\fun F \;\alt{\Varid{f}}{\Varid{g}}\comp \alt{\Varid{h}}{\Varid{k}}}
\just\equiv{ Fusão-\ensuremath{\mathbin{+}} (twice) }
\ensuremath{\alt{\Varid{out}\comp \Varid{f}}{\Varid{out}\comp \Varid{g}}\mathrel{=}\alt{\fun F \;\alt{\Varid{f}}{\Varid{g}}\comp \Varid{h}}{\fun F \;\alt{\Varid{f}}{\Varid{g}}\comp \Varid{k}}}
\just\equiv{ Eq-\ensuremath{\mathbin{+}} }
\ensuremath{\begin{lcbr}\Varid{out}\comp \Varid{f}\mathrel{=}\fun F \;\alt{\Varid{f}}{\Varid{g}}\comp \Varid{h}\\\Varid{out}\comp \Varid{g}\mathrel{=}\fun F \;\alt{\Varid{f}}{\Varid{g}}\comp \Varid{k}\end{lcbr}}
\end{eqnarray*}

De seguida iremos derivar um anamorfismo a partir do coproduto de \ensuremath{\Varid{h}} e \ensuremath{\Varid{k}}, definidos em \ensuremath{\Varid{fair\char95 merge}}:
\begin{eqnarray*}
\start
\ensuremath{\begin{lcbr}\Varid{h}\;(\Conid{Cons}\;(\Varid{x},\Varid{xs}),\Varid{y})\mathrel{=}\Conid{Cons}\;(\Varid{x},\Varid{k}\;(\Varid{xs},\Varid{y}))\\\Varid{k}\;(\Varid{x},\Conid{Cons}\;(\Varid{y},\Varid{ys}))\mathrel{=}\Conid{Cons}\;(\Varid{y},\Varid{h}\;(\Varid{x},\Varid{ys}))\end{lcbr}}
\just\equiv{ Def-\ensuremath{\times}, Def. \ensuremath{\Varid{assocr}}, Def. \ensuremath{\Varid{aux}} }
\ensuremath{\begin{lcbr}\Varid{h}\;((\Conid{Cons}\times\Varid{id})\;((\Varid{x},\Varid{xs}),\Varid{y}))\mathrel{=}\Conid{Cons}\;((\Varid{id}\times\Varid{k})\;(\Varid{assocr}\;((\Varid{x},\Varid{xs}),\Varid{y})))\\\Varid{k}\;((\Varid{id}\times\Conid{Cons})\;(\Varid{x},(\Varid{y},\Varid{ys})))\mathrel{=}\Conid{Cons}\;((\Varid{id}\times\Varid{h})\;(\Varid{aux}\;(\Varid{x},(\Varid{y},\Varid{ys}))))\end{lcbr}}
\just\equiv{ Def-comp, pointfree }
\ensuremath{\begin{lcbr}\Varid{h}\comp (\Conid{Cons}\times\Varid{id})\mathrel{=}\Conid{Cons}\comp (\Varid{id}\times\Varid{k})\comp \Varid{assocr}\\\Varid{k}\comp (\Varid{id}\times\Conid{Cons})\mathrel{=}\Conid{Cons}\comp (\Varid{id}\times\Varid{h})\comp \Varid{aux}\end{lcbr}}
\just\equiv{ ´Shunt-right´ }
\ensuremath{\begin{lcbr}\Varid{out}\comp \Varid{h}\comp (\Conid{Cons}\times\Varid{id})\mathrel{=}(\Varid{id}\times\Varid{k})\comp \Varid{assocr}\\\Varid{out}\comp \Varid{k}\comp (\Varid{id}\times\Conid{Cons})\mathrel{=}(\Varid{id}\times\Varid{h})\comp \Varid{aux}\end{lcbr}}
\just\impliedby{ Leibniz }
\ensuremath{\begin{lcbr}\Varid{out}\comp \Varid{h}\comp (\Conid{Cons}\times\Varid{id})\comp (\Varid{out}\times\Varid{id})\mathrel{=}(\Varid{id}\times\Varid{k})\comp \Varid{assocr}\comp (\Varid{out}\times\Varid{id})\\\Varid{out}\comp \Varid{k}\comp (\Varid{id}\times\Conid{Cons})\comp (\Varid{id}\times\Varid{out})\mathrel{=}(\Varid{id}\times\Varid{h})\comp \Varid{aux}\comp (\Varid{id}\times\Varid{out})\end{lcbr}}
\just\equiv{ isomorfismo in/out }
\ensuremath{\begin{lcbr}\Varid{out}\comp \Varid{h}\mathrel{=}(\Varid{id}\times\Varid{k})\comp \Varid{assocr}\comp (\Varid{out}\times\Varid{id})\\\Varid{out}\comp \Varid{k}\mathrel{=}(\Varid{id}\times\Varid{h})\comp \Varid{aux}\comp (\Varid{id}\times\Varid{out})\end{lcbr}}
\just\equiv{ Cancelamento-\ensuremath{\mathbin{+}} (twice) }
\ensuremath{\begin{lcbr}\Varid{out}\comp \Varid{h}\mathrel{=}(\Varid{id}\times(\alt{\Varid{h}}{\Varid{k}}\comp i_2))\comp \Varid{assocr}\comp (\Varid{out}\times\Varid{id})\\\Varid{out}\comp \Varid{k}\mathrel{=}(\Varid{id}\times(\alt{\Varid{h}}{\Varid{k}}\comp i_1))\comp \Varid{aux}\comp (\Varid{id}\times\Varid{out})\end{lcbr}}
\just\equiv{ Functor-\ensuremath{\times} }
\ensuremath{\begin{lcbr}\Varid{out}\comp \Varid{h}\mathrel{=}(\Varid{id}\times\alt{\Varid{h}}{\Varid{k}})\comp (\Varid{id}\times(i_2))\comp \Varid{assocr}\comp (\Varid{out}\times\Varid{id})\\\Varid{out}\comp \Varid{k}\mathrel{=}(\Varid{id}\times\alt{\Varid{h}}{\Varid{k}})\comp (\Varid{id}\times(i_1))\comp \Varid{aux}\comp (\Varid{id}\times\Varid{out})\end{lcbr}}
\just\equiv{ \ensuremath{\fun F \;\Varid{f}\mathrel{=}\Varid{id}\times\Varid{f}} }
\ensuremath{\begin{lcbr}\Varid{out}\comp \Varid{h}\mathrel{=}\fun F \;\alt{\Varid{h}}{\Varid{k}}\comp (\Varid{id}\times(i_2))\comp \Varid{assocr}\comp (\Varid{out}\times\Varid{id})\\\Varid{out}\comp \Varid{k}\mathrel{=}\fun F \;\alt{\Varid{h}}{\Varid{k}}\comp (\Varid{id}\times(i_1))\comp \Varid{aux}\comp (\Varid{id}\times\Varid{out})\end{lcbr}}
\just\equiv{ Lei dual da recursividade mútua }
\ensuremath{\alt{\Varid{h}}{\Varid{k}}\mathrel{=}\ana{\alt{(\Varid{id}\times(i_2))\comp \Varid{assocr}\comp (\Varid{out}\times\Varid{id})}{(\Varid{id}\times(i_1))\comp \Varid{aux}\comp (\Varid{id}\times\Varid{out})}}}
\end{eqnarray*}

\textbf{Nota}: Nos cálculos apresentados acima é referida uma função \ensuremath{\Varid{aux}}, esta função foi usada para simplificar a apresentação da função \ensuremath{\Varid{fair\char95 merge'}}. Podemos observar na definição de \ensuremath{\Varid{fair\char95 merge}}, mais precisamente na definição de \ensuremath{\Varid{k}}, que as instâncias \ensuremath{\Varid{x}} e \ensuremath{\Varid{y}} trocam de posições, logo a função \ensuremath{\Varid{aux}} efetua esta troca.

\begin{hscode}\SaveRestoreHook
\column{B}{@{}>{\hspre}l<{\hspost}@{}}%
\column{5}{@{}>{\hspre}l<{\hspost}@{}}%
\column{9}{@{}>{\hspre}l<{\hspost}@{}}%
\column{E}{@{}>{\hspre}l<{\hspost}@{}}%
\>[B]{}\Varid{fair\char95 merge'}\mathrel{=}\lanabracket\,\alt{\Varid{l}}{\Varid{r}}\,\ranabracket{}\<[E]%
\\
\>[B]{}\hsindent{5}{}\<[5]%
\>[5]{}\mathbf{where}{}\<[E]%
\\
\>[5]{}\hsindent{4}{}\<[9]%
\>[9]{}\Varid{aux}\;(\Varid{x},(\Varid{y},\Varid{z}))\mathrel{=}(\Varid{y},(\Varid{x},\Varid{z})){}\<[E]%
\\
\>[5]{}\hsindent{4}{}\<[9]%
\>[9]{}\Varid{l}\mathrel{=}(\Varid{id}\times(i_2))\comp \Varid{assocr}\comp (\mathsf{out}\times\Varid{id}){}\<[E]%
\\
\>[5]{}\hsindent{4}{}\<[9]%
\>[9]{}\Varid{r}\mathrel{=}(\Varid{id}\times(i_1))\comp \Varid{aux}\comp (\Varid{id}\times\mathsf{out}){}\<[E]%
\ColumnHook
\end{hscode}\resethooks


O diagrama do anamorfismo \ensuremath{\Varid{fair\char95 merge'}} é o seguinte:
\begin{eqnarray*}
\xymatrix@C=4cm@R=4em{
        \ensuremath{\Conid{Stream}\;\Conid{A}}
&
        \ensuremath{\Conid{A}\times\Conid{Stream}\;\Conid{A}}
                \ar[l]_-{\ensuremath{\Conid{Cons}}}
\\
        \ensuremath{(\Conid{Stream}\;\Conid{A})^2\mathbin{+}(\Conid{Stream}\;\Conid{A})^2}
                \ar[u]^-{\ensuremath{\lanabracket\,\alt{\Varid{l}}{\Varid{r}}\,\ranabracket}}
                \ar[r]_-{\ensuremath{\alt{\Varid{l}}{\Varid{r}}}}
&
        \ensuremath{\Conid{A}\times(\Conid{Stream}\;\Conid{A})^2\mathbin{+}(\Conid{Stream}\;\Conid{A})^2}
                \ar[u]_-{\ensuremath{\Varid{id}\mathbin{+}\lanabracket\,\alt{\Varid{l}}{\Varid{r}}\,\ranabracket}}
}
\end{eqnarray*}

O diagrama da alternativa \ensuremath{\alt{\Varid{l}}{\Varid{r}}} é o seguinte:
\begin{eqnarray*}
\xymatrix{
        \ensuremath{(\Conid{Stream}\;\Conid{A})^2}
                \ar[r]^-{\ensuremath{i_1}}
                \ar[d]_-{\ensuremath{\Varid{out}\times\Varid{id}}}
&
        \ensuremath{(\Conid{Stream}\;\Conid{A})^2\mathbin{+}(\Conid{Stream}\;\Conid{A})^2}
                \ar[ddd]_{\ensuremath{\alt{\Varid{l}}{\Varid{r}}}}
&
        \ensuremath{(\Conid{Stream}\;\Conid{A})^2}
                \ar[l]_-{\ensuremath{i_2}}
                \ar[d]^-{\ensuremath{\Varid{id}\times\Varid{out}}}
\\
        \ensuremath{(\Conid{A}\times\Conid{Stream}\;\Conid{A})\times\Conid{Stream}\;\Conid{A}}
                \ar[d]_{\ensuremath{\Varid{assocr}}}
&
&
        \ensuremath{\Conid{Stream}\;\Conid{A}\times(\Conid{A}\times\Conid{Stream}\;\Conid{A})}
                \ar[d]^{\ensuremath{\Varid{aux}}}
\\
        \ensuremath{\Conid{A}\times(\Conid{Stream}\;\Conid{A})^2}
                \ar[dr]^{\ensuremath{\Varid{id}\times(i_2)}}
&
&
        \ensuremath{\Conid{A}\times(\Conid{Stream}\;\Conid{A})^2}
                \ar[dl]_{\ensuremath{\Varid{id}\times(i_1)}}
\\
&
        \ensuremath{\Conid{A}\times(\Conid{Stream}\;\Conid{A})^2\mathbin{+}(\Conid{Stream}\;\Conid{A})^2}
&
}
\end{eqnarray*}

De forma a garantir que o comportamento da função \ensuremath{\Varid{fair\char95 merge'}} é realmente o correto, definimos as seguintes instâncias de \ensuremath{\Conid{Stream}\;\Conid{Int}} e a função \ensuremath{\Varid{takeStream}}, dado que estamos perante um tipo de dados infinito, podemos tirar partido da \lazy{lazy evalution} do \Haskell.
\begin{hscode}\SaveRestoreHook
\column{B}{@{}>{\hspre}l<{\hspost}@{}}%
\column{E}{@{}>{\hspre}l<{\hspost}@{}}%
\>[B]{}\Varid{ones}\mathbin{::}\Conid{Stream}\;\Conid{Int}{}\<[E]%
\\
\>[B]{}\Varid{ones}\mathrel{=}\Conid{Cons}\;(\mathrm{1},\Varid{ones}){}\<[E]%
\\[\blanklineskip]%
\>[B]{}\Varid{twos}\mathbin{::}\Conid{Stream}\;\Conid{Int}{}\<[E]%
\\
\>[B]{}\Varid{twos}\mathrel{=}\Conid{Cons}\;(\mathrm{2},\Varid{twos}){}\<[E]%
\\[\blanklineskip]%
\>[B]{}\Varid{takeStream}\mathbin{::}\Conid{Int}\to \Conid{Stream}\;\Varid{a}\to [\mskip1.5mu \Varid{a}\mskip1.5mu]{}\<[E]%
\\
\>[B]{}\Varid{takeStream}\;\mathrm{0}\;\anonymous \mathrel{=}[\mskip1.5mu \mskip1.5mu]{}\<[E]%
\\
\>[B]{}\Varid{takeStream}\;(\Varid{n}\mathbin{+}\mathrm{1})\;(\Conid{Cons}\;(\Varid{x},\Varid{s}))\mathrel{=}\Varid{x}\mathbin{:}\Varid{takeStream}\;\Varid{n}\;\Varid{s}{}\<[E]%
\ColumnHook
\end{hscode}\resethooks

Podemos usar esta função para verificar se \ensuremath{\Varid{fair\char95 merge}} e \ensuremath{\Varid{fair\char95 merge'}} têm o mesmo comportamento, mas é importante realçar que esta demonstração não é uma prova total.
\begin{hscode}\SaveRestoreHook
\column{B}{@{}>{\hspre}l<{\hspost}@{}}%
\column{16}{@{}>{\hspre}l<{\hspost}@{}}%
\column{20}{@{}>{\hspre}l<{\hspost}@{}}%
\column{E}{@{}>{\hspre}l<{\hspost}@{}}%
\>[B]{}\Varid{streamTest}\;\Varid{n}\mathrel{=}\mathbf{let}{}\<[E]%
\\
\>[B]{}\hsindent{20}{}\<[20]%
\>[20]{}\Varid{input}\mathrel{=}i_1\;(\Varid{ones},\Varid{twos}){}\<[E]%
\\
\>[B]{}\hsindent{20}{}\<[20]%
\>[20]{}\Varid{l}\mathrel{=}\Varid{takeStream}\;\Varid{n}\;(\Varid{fair\char95 merge}\;\Varid{input}){}\<[E]%
\\
\>[B]{}\hsindent{20}{}\<[20]%
\>[20]{}\Varid{r}\mathrel{=}\Varid{takeStream}\;\Varid{n}\;(\Varid{fair\char95 merge'}\;\Varid{input}){}\<[E]%
\\
\>[B]{}\hsindent{16}{}\<[16]%
\>[16]{}\mathbf{in}\;\Varid{l}\equiv \Varid{r}{}\<[E]%
\ColumnHook
\end{hscode}\resethooks


\subsection*{Problema 4}

Para resolver este problema, iremos dividir a resolução em duas fases: (1) definir o operador \ensuremath{\Varid{pcataList}} e (2) definir \ensuremath{\Varid{gene}}. Podemos verificar que os tipos de \ensuremath{\Varid{pcataList}} e \ensuremath{\llparenthesis\, \cdot \,\rrparenthesis} são semelhantes, e numa primeira tentativa assumir que são iguais.
\begin{eqnarray*}
\xymatrix@C=3cm@R=2cm{
        \ensuremath{{\Conid{A}}^{*}}
                \ar[d]_-{\ensuremath{\Varid{pcataList}\;\Varid{g}}}
                \ar@/^1pc/[r]^-{\ensuremath{\Varid{outList}}}
&
        \ensuremath{\mathrm{1}\mathbin{+}\Conid{A}\times{\Conid{A}}^{*}}
                \ar[d]^{\ensuremath{\Varid{id}\mathbin{+}\Varid{id}\times\Varid{pcataList}\;\Varid{g}}}
                \ar@/^1pc/[l]^-{\ensuremath{\Varid{inList}}}
\\
        \ensuremath{\fun{Dist}\;\mathit B}
&
        \ensuremath{\mathrm{1}\mathbin{+}\Conid{A}\times\fun{Dist}\;\mathit B}
                \ar[l]^-{\ensuremath{\Varid{g}}}
}
\end{eqnarray*}

Analisando o diagrama verificamos que o tipo do gene \ensuremath{\Varid{g}} não encaixa no diagrama, pois o seu tipo é \ensuremath{\Varid{g}\mathbin{:}\mathrm{1}\mathbin{+}\Conid{A}\times\mathit B\to \fun{Dist}\;\mathit B}, logo não poderemos definir \ensuremath{\Varid{pcataList}} como um catamorfismo de listas "normal". (É importante salientar que o diagrama apresentado é \textbf{inválido}, isto é, os tipos não são válidos.)

Embora o diagrama seja inválido, podemos tentar corrigi-lo através da definição de algumas funções, isto é, se formos capazes de definir uma composição de funções que satisfaça o tipo \ensuremath{\mathrm{1}\mathbin{+}\Conid{A}\times\fun{Dist}\;\mathit B\to \fun{Dist}\;\mathit B} e que envolva o gene \ensuremath{\Varid{g}}.

Sendo \ensuremath{\fun{Dist}\;\mathit B} um Monad, como podemos verificar na biblioteca \Probability\ através da definição de \ensuremath{\Varid{return}} e de \ensuremath{\bind }:
\begin{itemize}
     \item \ensuremath{\Varid{return}\;\Varid{x}\mathrel{=}\Conid{D}\;[\mskip1.5mu (\Varid{x},\mathrm{1})\mskip1.5mu]}
     \item \ensuremath{\Varid{d}\bind \Varid{f}\mathrel{=}\Conid{D}\;[\mskip1.5mu (\Varid{y},\Varid{q}\mathbin{*}\Varid{p})\mid (\Varid{x},\Varid{p})\leftarrow \Varid{unD}\;\Varid{d},(\Varid{y},\Varid{q})\leftarrow \Varid{unD}\;(\Varid{f}\;\Varid{x})\mskip1.5mu]}), e o gene \ensuremath{\Varid{g}\mathbin{:}\mathrm{1}\mathbin{+}\Conid{A}\times\mathit B\to \fun{Dist}\;\mathit B}
\end{itemize}

Podemos tentar definir uma composição monádica \ensuremath{\Varid{g}\kcomp \Varid{f}\comp \fun F \;(\Varid{pcataList}\;\Varid{g})\comp \Varid{outList}}, com \ensuremath{\Varid{f}\mathbin{:}\mathrm{1}\mathbin{+}\Conid{A}\times\fun{Dist}\;\mathit B\to \fun{Dist}\;(\mathrm{1}\mathbin{+}\Conid{A}\times\mathit B)}. Conseguimos deduzir que \ensuremath{\Varid{f}} será uma alternativa, dado o seu tipo ser do género \ensuremath{\Conid{A}\mathbin{+}\mathit B\to \mathit C}. O lado esquerdo da alternativa é simples de resolver, bastando efetuar uma injeção e colocar o resultado na "caixa" \ensuremath{\fun{Dist}}, ou seja \ensuremath{\Varid{f}\mathrel{=}\alt{\Varid{return}\comp i_1}{\bot }}. Para o lado direito da alternativa teremos que emparelhar cada elemento \ensuremath{\Varid{b}} que está dentro do Monad \ensuremath{\fun{Dist}\;\mathit B} com o elemento \ensuremath{\Conid{A}}. Tal é possível através da notação \ensuremath{\mathbf{do}} em Haskell, que nos permite "iterar" sobre os elementos de um Monad. Definimos então a função \ensuremath{\Varid{attach}} que pega num elemento puro e combina-o dentro de um Monad.
\begin{hscode}\SaveRestoreHook
\column{B}{@{}>{\hspre}l<{\hspost}@{}}%
\column{E}{@{}>{\hspre}l<{\hspost}@{}}%
\>[B]{}\Varid{attach}\mathbin{::}(\Conid{Monad}\;\Varid{m})\Rightarrow (\Varid{a},\Varid{m}\;\Varid{b})\to \Varid{m}\;(\Varid{a},\Varid{b}){}\<[E]%
\\
\>[B]{}\Varid{attach}\;(\Varid{b},\Varid{x})\mathrel{=}\mathbf{do}\;\{\mskip1.5mu \Varid{a}\leftarrow \Varid{x};\Varid{return}\;(\Varid{b},\Varid{a})\mskip1.5mu\}{}\<[E]%
\ColumnHook
\end{hscode}\resethooks

Por fim basta injetar cada elemento de \ensuremath{\fun{Dist}\;(\Conid{A}\times\mathit B)} através de \ensuremath{\mathsf{fmap}\;i_2}:
\begin{hscode}\SaveRestoreHook
\column{B}{@{}>{\hspre}l<{\hspost}@{}}%
\column{E}{@{}>{\hspre}l<{\hspost}@{}}%
\>[B]{}\Varid{ddist}\mathrel{=}\alt{\Varid{return}\comp i_1}{\mathsf{fmap}\;i_2\comp \Varid{attach}}{}\<[E]%
\ColumnHook
\end{hscode}\resethooks

Podemos então apresentar o diagrama de \ensuremath{\Varid{ddist}}:
\begin{eqnarray*}
\xymatrix@C=2cm@R=2cm{
        \ensuremath{\mathrm{1}}
                \ar[r]^-{\ensuremath{i_1}}
                \ar[d]_-{\ensuremath{i_1}}
&
        \ensuremath{\mathrm{1}\mathbin{+}\Conid{A}\times\fun{Dist}\;\mathit B}
                \ar[d]_{\ensuremath{\Varid{ddist}}}
&
        \ensuremath{\Conid{A}\times\fun{Dist}\;\mathit B}
                \ar[l]_-{\ensuremath{i_2}}
                \ar[d]^-{\ensuremath{\Varid{attach}}}
\\
        \ensuremath{\mathrm{1}\mathbin{+}\Conid{A}\times\mathit B}
                \ar[r]_{\ensuremath{\Varid{return}}}
&
        \ensuremath{\fun{Dist}\;(\mathrm{1}\mathbin{+}\Conid{A}\times\mathit B)}
&
        \ensuremath{\fun{Dist}\;(\Conid{A}\times\mathit B)}
                \ar[l]^{\ensuremath{\mathsf{fmap}\;i_2}}
}
\end{eqnarray*}

Por fim, podemos definir o operador \ensuremath{\Varid{pcataList}} através da composição monádica descrita anteriormente:
\begin{hscode}\SaveRestoreHook
\column{B}{@{}>{\hspre}l<{\hspost}@{}}%
\column{E}{@{}>{\hspre}l<{\hspost}@{}}%
\>[B]{}\Varid{pcataList}\;\Varid{g}\mathrel{=}\Varid{g}\mathbin{.!}(\Varid{ddist}\comp \Varid{recList}\;(\Varid{pcataList}\;\Varid{g})\comp \Varid{outList}){}\<[E]%
\ColumnHook
\end{hscode}\resethooks

O diagrama da composição monádica é o seguinte (abreviando \ensuremath{\Varid{f}\mathrel{=}\Varid{ddist}\comp \Varid{recList}\;(\Varid{pcataList}\;\Varid{g})\comp \Varid{outList}}):
\begin{eqnarray*}
\xymatrix@C=5em{
    \ensuremath{\fun{Dist}\;(\fun{Dist}\;\mathit B)}
        \ar[d]_{\ensuremath{\mu }}
&
    \ensuremath{\fun{Dist}\;(\mathrm{1}\mathbin{+}\Conid{A}\times\mathit B)}
        \ar[l]_{\ensuremath{\Conid{T}\;\Varid{g}}}
        \ar@{..}[d]
&
    \ensuremath{{\Conid{A}}^{*}}
        \ar[l]_{\ensuremath{\Varid{f}}}
        \ar@/^3pc/[dll]^-{\ensuremath{\Varid{g}\kcomp \Varid{f}}}
\\
    \ensuremath{\fun{Dist}\;\mathit B}
&
    \ensuremath{\mathit B}
        \ar[l]_{\ensuremath{\Varid{g}}}
&
\\
&
&
&
}
\end{eqnarray*}

Apresentamos então o diagrama de \ensuremath{\Varid{pacataList}}:
\begin{eqnarray*}
\xymatrix@C=4em@R=3em{
    \ensuremath{{\Conid{A}}^{*}}
           \ar@/_4pc/[ddd]_{\ensuremath{\Varid{pcataList}\;\Varid{g}}}
           \ar@{..}[dd]
           \ar[r]^{\ensuremath{\Varid{outList}}}
&
    \ensuremath{\mathrm{1}\mathbin{+}\Conid{A}\times{\Conid{A}}^{*}}
           \ar[d]^{\ensuremath{\Varid{id}\mathbin{+}\Varid{id}\times\Varid{pcataList}\;\Varid{g}}}
\\
&
     \ensuremath{\mathrm{1}\mathbin{+}\Conid{A}\times\fun{Dist}\;\mathit B}
           \ar[d]^{\ensuremath{\Varid{ddist}}}
\\
     \ensuremath{\fun{Dist}\;(\fun{Dist}\;\mathit B)}
           \ar[d]^{\ensuremath{\mu }}
&
     \ensuremath{\fun{Dist}\;(\mathrm{1}\mathbin{+}\Conid{A}\times\mathit B)}
           \ar[l]^{\ensuremath{\Conid{T}\;\Varid{g}}}
           \ar@{..}[d]
\\
     \ensuremath{\fun{Dist}\;\mathit B}
&
     \ensuremath{\mathrm{1}\mathbin{+}\Conid{A}\times\mathit B}
           \ar[l]^{\ensuremath{\Varid{g}}}
}
\end{eqnarray*}

Na segunda fase deste problema temos de definir \ensuremath{\Varid{gene}}, que simula o comportamento descrito no exemplo. Esta função será uma alternativa entre duas funções, uma delas que trabalha sobre o único habitante to tipo \ensuremath{\mathrm{1}}, que corresponde a uma lista vazia neste caso, ou seja, o fim da mensagem. A palavra "stop" deverá ser transmitida no fim de cada mensagem, com $10\%$ de probabilidade de falhar, logo terá $90\%$ de ser enviada. Seguindo a mesma estratégia da função apresentada \ensuremath{\Varid{padd}}, podemos definir \ensuremath{\Varid{pstop}} da seguinte forma:
\begin{hscode}\SaveRestoreHook
\column{B}{@{}>{\hspre}l<{\hspost}@{}}%
\column{E}{@{}>{\hspre}l<{\hspost}@{}}%
\>[B]{}\Varid{pstop}\;\anonymous \mathrel{=}\Conid{D}\;[\mskip1.5mu ([\mskip1.5mu \text{\ttfamily \char34 stop\char34}\mskip1.5mu],\mathrm{0.9}),([\mskip1.5mu \mskip1.5mu],\mathrm{0.1})\mskip1.5mu]{}\<[E]%
\ColumnHook
\end{hscode}\resethooks

Sabemos ainda que a probabilidade de uma palavra ser enviada é de $95\%$, dado que a probabilidade de se perder é de $5\%$. Logo, considerando uma mensagem como uma lista de palavras não vazia, a função que descreve este comportamento é a seguinte:
\begin{hscode}\SaveRestoreHook
\column{B}{@{}>{\hspre}l<{\hspost}@{}}%
\column{E}{@{}>{\hspre}l<{\hspost}@{}}%
\>[B]{}\Varid{plose}\;(\Varid{w},\Varid{xs})\mathrel{=}\Conid{D}\;[\mskip1.5mu (\Varid{w}\mathbin{:}\Varid{xs},\mathrm{0.95}),(\Varid{xs},\mathrm{0.05})\mskip1.5mu]{}\<[E]%
\ColumnHook
\end{hscode}\resethooks

Definidas as funções que tratam do fim e do corpo de uma mensagem, podemos então definir \ensuremath{\Varid{gene}}, como a alternativa entre estas:
\begin{hscode}\SaveRestoreHook
\column{B}{@{}>{\hspre}l<{\hspost}@{}}%
\column{E}{@{}>{\hspre}l<{\hspost}@{}}%
\>[B]{}\Varid{gene}\mathrel{=}\alt{\Varid{pstop}}{\Varid{plose}}{}\<[E]%
\ColumnHook
\end{hscode}\resethooks

Através de \ensuremath{\Varid{pcataList}} e \ensuremath{\Varid{gene}}, podemos ver quais são as probabilidade associadas à transmissão da mensagem \ensuremath{\Varid{words}\;\text{\ttfamily \char34 Vamos~atacar~hoje\char34}}, através da execução de \ensuremath{\Varid{transmitir}}:
\begin{Verbatim}[fontsize=\small]
["Vamos","atacar","hoje","stop"]  77.2%
       ["Vamos","atacar","hoje"]   8.6%
        ["atacar","hoje","stop"]   4.1%
       ["Vamos","atacar","stop"]   4.1%
         ["Vamos","hoje","stop"]   4.1%
              ["Vamos","atacar"]   0.5%
                ["Vamos","hoje"]   0.5%
               ["atacar","hoje"]   0.5%
                ["Vamos","stop"]   0.2%
               ["atacar","stop"]   0.2%
                 ["hoje","stop"]   0.2%
                      ["atacar"]   0.0%
                       ["Vamos"]   0.0%
                        ["hoje"]   0.0%
                        ["stop"]   0.0%
                              []   0.0%
\end{Verbatim}

Vamos então responder às perguntas:
\begin{itemize}
\item Qual a probabilidade de o resultado da transmissão ser \ensuremath{[\mskip1.5mu \text{\ttfamily \char34 Vamos\char34},\text{\ttfamily \char34 hoje\char34},\text{\ttfamily \char34 stop\char34}\mskip1.5mu]}?
Terá uma probabilidade de $4.1\%$.
\item Qual a probabilidade de seguirem todas as palavras, mas faltar "stop" no fim?
Terá uma probabilidade de $8.6\%$.
\item Qual a probabilidade de o resultado da transmissão ser perfeito?
Terá uma probabilidade de $77.2\%$.
\end{itemize}

De forma a garantir que estamos a lidar com uma distribuição válida, vamos recorrer às funções \ensuremath{\Varid{checkD}} e \ensuremath{\Varid{sumP}} definidas na biblioteca \Probability\, que validam uma distribuição e determinam o somatório das probabilidades, respetivamente, para verificar a sua validade. Através da execução da seguinte equação:
\begin{hscode}\SaveRestoreHook
\column{B}{@{}>{\hspre}l<{\hspost}@{}}%
\column{E}{@{}>{\hspre}l<{\hspost}@{}}%
\>[B]{}\Varid{checkedD}\mathrel{=}\Varid{checkD}\;(\Varid{transmitir}\;(\Varid{words}\;\text{\ttfamily \char34 Vamos~atacar~hoje\char34})){}\<[E]%
\ColumnHook
\end{hscode}\resethooks

Verificamos que \ensuremath{\Varid{checkedD}} não apresenta nenhuma mensagem de erro e tem o mesmo valor que \ensuremath{\Varid{transmitir}\;(\Varid{words}\;\text{\ttfamily \char34 Vamos~atacar~hoje\char34})}.
\begin{hscode}\SaveRestoreHook
\column{B}{@{}>{\hspre}l<{\hspost}@{}}%
\column{E}{@{}>{\hspre}l<{\hspost}@{}}%
\>[B]{}\Varid{valueD}\mathrel{=}(\Varid{sumP}\comp \Varid{unD}\comp \Varid{transmitir}\comp \Varid{words})\;\text{\ttfamily \char34 Vamos~atacar~hoje\char34}{}\<[E]%
\ColumnHook
\end{hscode}\resethooks

Provamos também que \ensuremath{\Varid{valueD}\mathrel{=}\mathrm{1.0}}, logo estamos perante uma distribuição válida.


%----------------- Índice remissivo (exige makeindex) -------------------------%

\printindex

%----------------- Bibliografia (exige bibtex) --------------------------------%

\bibliographystyle{plain}
\bibliography{cp2526t}

%----------------- Fim do documento -------------------------------------------%
\end{document}
